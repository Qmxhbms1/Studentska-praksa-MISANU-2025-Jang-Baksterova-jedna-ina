\documentclass{article}

\input{homework-preamble.tex}

\title{Yang-Baxter-like matrix equation}
\author{Mihailo Đurić}
\date{\today}

\begin{document}

\maketitle
\newpage

\begin{problem}
  Let
  \[A = \begin{bmatrix} 0 & 1 & 0\\ 1 & 1 & 0\\ 1 & 0 & 0 \end{bmatrix}.\]
  \begin{itemize}
    \item Find $P$ and $Q$ such that $PA = AQ = 0$.
    \item Find some inner inverse of $A$.
    \item Find one solution to YBME for this particular $A$.
      Try to generate some new classes of solutions with it.
  \end{itemize}
\end{problem}

\begin{solution}
  We are looking for a matrix $P = \begin{bmatrix} a_{11} & a_{12} & a_{13}\\ a_{21} & a_{22} & a_{23}\\ a_{31} & a_{32} & a_{33} \end{bmatrix}$ such that
  \[PA = \begin{bmatrix} a_{11} & a_{12} & a_{13}\\ a_{21} & a_{22} & a_{23}\\ a_{31} & a_{32} & a_{33} \end{bmatrix} \cdot \begin{bmatrix} 0 & 1 & 0\\ 1 & 1 & 0\\ 1 & 0 & 0 \end{bmatrix} = \begin{bmatrix} a_{12} + a_{13} & a_{11} + a_{12} & 0\\ a_{22} + a_{23} & a_{21} + a_{22} & 0\\ a_{32} + a_{33} & a_{31} + a_{32} & 0 \end{bmatrix} = 0.\]
  Thus we have $a_{i1} = - a_{i2} = a_{i3}$ for $i \in \{1, 2, 3\}$.
  Choosing $a_{i1} = 1$ we get
  \[P = \begin{bmatrix} 1 & -1 & 1\\ 1 & -1 & 1\\ 1 & -1 & 1 \end{bmatrix}.\]
  Similarly we get that $Q$ is of the form
  \[\begin{bmatrix} 0 & 0 & 0\\ 0 & 0 & 0\\ a_{31} & a_{32} & a_{33} \end{bmatrix},\]
  for any $a_{3i} \in \C$.
  Thus we can simply choose
  \[Q = \begin{bmatrix} 0 & 0 & 0\\ 0 & 0 & 0\\ 1 & 1 & 1 \end{bmatrix}.\]
  Thus we have have $PA = QA = 0$.

  We put
  \[B = \begin{bmatrix} a_{11} & a_{12} & a_{13}\\ a_{21} & a_{22} & a_{23}\\ a_{31} & a_{32} & a_{33} \end{bmatrix}.\]
  Then we have
  \[ABA = \begin{bmatrix} a_{22} + a_{23} & a_{21} + a_{22} & 0\\ a_{12} + a_{13} + a_{22} + a_{23} & a_{11} + a_{12} + a_{21} + a_{22} & 0\\ a_{12} + a_{13} & a_{11} + a_{12} & 0 \end{bmatrix} = A.\]
  Thus we have $a_{22} = - a_{23} = 1 - a_{21}$, $a_{12} = 1 - a_{13} = - a_{11}$.
  Picking $a_{22} = a_{12} = 1$ and $a_{3j} = 0$ we have
  \[B = \begin{bmatrix} -1 & 1 & 0\\ 0 & 1 & -1\\ 0 & 0 & 0 \end{bmatrix},\]
  and $B$ is an inner inverse of $A$.

  We already know that $P$ and $Q$ are solutions to YBME, thus from the Lemma on page 6 we know that any powers of $P$ and $Q$ are also solutions.
  Further more, we know that any matrix $Q^n P^m$ is a solution for any $n, m \in \N$.
  We can also insert $B$ anywhere in that solution by the theorem on page 7, so we have $Q^{n - i} B Q^i P^m$ and $Q^n P^j B P^{m - j}$ as families of solutions.
  We can continue inserting $B$, either raising its power or inserting it in a different place now to generate new solutions.
  We can also easily generate new matrices $P$, $Q$ and $B$ and repeat the process, or mix the solutions.
\end{solution}

\begin{problem}
  A square matrix $L$ is nilpotent if and only if $\sigma(L) = \{0\}$.
\end{problem}

\begin{solution}
  First we assume that $L$ is nilpotent, i.e., that there exists some $n \in \N$ such that $L^n = 0$.
  We consider the polynomial $p(A) = A^n$.
  We clearly see that $p(L) = 0$, thus $\sigma(p(L)) = 0$.
  By the spectral mapping theorem we have $p(\sigma(L)) = 0$, so for any eigenvalue $\lambda$ of $L$ we know that $\lambda^n = 0$.
  From here we clearly have $\lambda = 0$.
  
  Conversely, assume that $\sigma(L) = \{0\}$.
\end{solution}

\end{document}
