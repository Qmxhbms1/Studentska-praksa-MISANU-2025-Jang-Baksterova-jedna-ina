\documentclass{article}

% ===================================================================
% PREAMBLE FOR MATHEMATICS EXERCISE NOTEBOOK
% ===================================================================
% This file contains all the packages and custom commands for the document.
% Keeping it separate from main.tex makes the project cleaner.


% -------------------------------------------------------------------
% DOCUMENT & ENCODING
% -------------------------------------------------------------------
\usepackage[utf8]{inputenc} % Allows you to type characters like á, ö, etc. directly
\usepackage[T1]{fontenc}    % Specifies font encoding, improves font rendering and hyphenation


% -------------------------------------------------------------------
% PAGE LAYOUT & GEOMETRY
% -------------------------------------------------------------------
\usepackage{geometry}
\geometry{
  a4paper,         % Or letterpaper, etc.
  total={170mm,257mm},
  left=20mm,
  top=20mm,
}


% -------------------------------------------------------------------
% CORE MATH PACKAGES (AMS - American Mathematical Society)
% -------------------------------------------------------------------
\usepackage{amsmath}    % The premier package for typesetting math equations
\usepackage{amssymb}    % Provides lots of extra math symbols (like \mathbb)
\usepackage{amsthm}     % Comprehensive theorem-like environments
\usepackage{mathtools}  % An extension of amsmath, provides more tools and fixes


% -------------------------------------------------------------------
% UTILITY & FORMATTING PACKAGES
% -------------------------------------------------------------------
\usepackage{graphicx}      % For including images (\includegraphics)
\usepackage{xcolor}        % For defining and using colors
\usepackage{booktabs}      % For creating beautiful, professional-looking tables (\toprule, \midrule, \bottomrule)
\usepackage{siunitx}       % For typesetting numbers and units consistently
\usepackage{enumitem}      % Provides more control over list environments (itemize, enumerate)


% -------------------------------------------------------------------
% HYPERLINKS & CROSS-REFERENCING
% -------------------------------------------------------------------
\usepackage{hyperref}
\hypersetup{
    colorlinks=true,       % false: boxed links; true: colored links
    linkcolor=teal,        % color of internal links (e.g., sections)
    citecolor=green,       % color of links to bibliography
    filecolor=magenta,     % color of file links
    urlcolor=blue          % color of external links
}


% -------------------------------------------------------------------
% THEOREM & DEFINITION ENVIRONMENTS
% -------------------------------------------------------------------
% This section sets up consistent numbering and styling for problems,
% definitions, theorems, etc.

\theoremstyle{definition} % Use a style that is less flashy than the default "plain" style
\newtheorem{problem}{Problem} % Number problems as Problem X.Y (where X is the section number)
\newtheorem{theorem}[problem]{Theorem}     % Share the same counter as 'problem'
\newtheorem{lemma}[problem]{Lemma}
\newtheorem{corollary}[problem]{Corollary}
\newtheorem{definition}[problem]{Definition}
\newtheorem{example}[problem]{Example}
\newtheorem{remark}[problem]{Remark}


% -------------------------------------------------------------------
% CUSTOM SOLUTION ENVIRONMENT
% -------------------------------------------------------------------
% We create a new "solution" environment that looks clean and
% automatically adds a QED symbol at the end.

\newenvironment{solution}
  {
   % Begin environment
   \renewcommand\qedsymbol{$\blacksquare$} % Change the default QED symbol to a black square
   \begin{proof}[Solution] % Base it on the 'proof' environment but change the title
  }
  {
   % End environment
   \end{proof}
  }


% -------------------------------------------------------------------
% CUSTOM MATH COMMANDS (MACROS)
% -------------------------------------------------------------------
% Define shortcuts for commonly used mathematical notation to save
% time and ensure consistency.

% Sets of numbers
\newcommand{\R}{\mathbb{R}} % Real numbers
\newcommand{\C}{\mathbb{C}} % Complex numbers
\newcommand{\N}{\mathbb{N}} % Natural numbers
\newcommand{\Z}{\mathbb{Z}} % Integers
\newcommand{\Q}{\mathbb{Q}} % Rational numbers

% Calculus operators
\newcommand{\dd}{\, \mathrm{d}} % For integrals, e.g., \int f(x)\dd x
\newcommand{\pdv}[2]{\frac{\partial #1}{\partial #2}} % Partial derivative
\newcommand{\dv}[2]{\frac{\mathrm{d} #1}{\mathrm{d} #2}} % Full derivative

% Linear Algebra
\DeclareMathOperator{\Tr}{Tr} % Trace of a matrix
\newcommand{\T}{\mathsf{T}}   % Transpose, e.g., A^\T

% Probability & Statistics
\DeclareMathOperator{\E}{\mathbb{E}} % Expectation
\DeclareMathOperator{\Var}{Var} % Variance
\DeclareMathOperator{\Cov}{Cov} % Covariance


% ===================================================================
% END OF PREAMBLE
% ===================================================================


\title{Yang-Baxter-like matrix equation}
\author{Mihailo Đurić}
\date{\today}

\begin{document}

\maketitle
\newpage

\begin{problem}
  Let
  \[A = \begin{bmatrix} 0 & 1 & 0\\ 1 & 1 & 0\\ 1 & 0 & 0 \end{bmatrix}.\]
  \begin{itemize}
    \item Find $P$ and $Q$ such that $PA = AQ = 0$.
    \item Find some inner inverse of $A$.
    \item Find one solution to YBME for this particular $A$.
      Try to generate some new classes of solutions with it.
  \end{itemize}
\end{problem}

\begin{solution}
  We are looking for a matrix $P = \begin{bmatrix} a_{11} & a_{12} & a_{13}\\ a_{21} & a_{22} & a_{23}\\ a_{31} & a_{32} & a_{33} \end{bmatrix}$ such that
  \[PA = \begin{bmatrix} a_{11} & a_{12} & a_{13}\\ a_{21} & a_{22} & a_{23}\\ a_{31} & a_{32} & a_{33} \end{bmatrix} \cdot \begin{bmatrix} 0 & 1 & 0\\ 1 & 1 & 0\\ 1 & 0 & 0 \end{bmatrix} = \begin{bmatrix} a_{12} + a_{13} & a_{11} + a_{12} & 0\\ a_{22} + a_{23} & a_{21} + a_{22} & 0\\ a_{32} + a_{33} & a_{31} + a_{32} & 0 \end{bmatrix} = 0.\]
  Thus we have $a_{i1} = - a_{i2} = a_{i3}$ for $i \in \{1, 2, 3\}$.
  Choosing $a_{i1} = 1$ we get
  \[P = \begin{bmatrix} 1 & -1 & 1\\ 1 & -1 & 1\\ 1 & -1 & 1 \end{bmatrix}.\]
  Similarly we get that $Q$ is of the form
  \[\begin{bmatrix} 0 & 0 & 0\\ 0 & 0 & 0\\ a_{31} & a_{32} & a_{33} \end{bmatrix},\]
  for any $a_{3i} \in \C$.
  Thus we can simply choose
  \[Q = \begin{bmatrix} 0 & 0 & 0\\ 0 & 0 & 0\\ 1 & 1 & 1 \end{bmatrix}.\]
  Thus we have have $PA = QA = 0$.

  We put
  \[B = \begin{bmatrix} a_{11} & a_{12} & a_{13}\\ a_{21} & a_{22} & a_{23}\\ a_{31} & a_{32} & a_{33} \end{bmatrix}.\]
  Then we have
  \[ABA = \begin{bmatrix} a_{22} + a_{23} & a_{21} + a_{22} & 0\\ a_{12} + a_{13} + a_{22} + a_{23} & a_{11} + a_{12} + a_{21} + a_{22} & 0\\ a_{12} + a_{13} & a_{11} + a_{12} & 0 \end{bmatrix} = A.\]
  Thus we have $a_{22} = - a_{23} = 1 - a_{21}$, $a_{12} = 1 - a_{13} = - a_{11}$.
  Picking $a_{22} = a_{12} = 1$ and $a_{3j} = 0$ we have
  \[B = \begin{bmatrix} -1 & 1 & 0\\ 0 & 1 & -1\\ 0 & 0 & 0 \end{bmatrix},\]
  and $B$ is an inner inverse of $A$.

  We already know that $P$ and $Q$ are solutions to YBME, thus from the Lemma on page 6 we know that any powers of $P$ and $Q$ are also solutions.
  Further more, we know that any matrix $Q^n P^m$ is a solution for any $n, m \in \N$.
  We can also insert $B$ anywhere in that solution by the theorem on page 7, so we have $Q^{n - i} B Q^i P^m$ and $Q^n P^j B P^{m - j}$ as families of solutions.
  We can continue inserting $B$, either raising its power or inserting it in a different place now to generate new solutions.
  We can also easily generate new matrices $P$, $Q$ and $B$ and repeat the process, or mix the solutions.
\end{solution}

\begin{problem}
  A square matrix $L$ is nilpotent if and only if $\sigma(L) = \{0\}$.
\end{problem}

\begin{solution}
  First we assume that $L$ is nilpotent, i.e., that there exists some $n \in \N$ such that $L^n = 0$.
  We consider the polynomial $p(A) = A^n$.
  We clearly see that $p(L) = 0$, thus $\sigma(p(L)) = 0$.
  By the spectral mapping theorem we have $p(\sigma(L)) = 0$, so for any eigenvalue $\lambda$ of $L$ we know that $\lambda^n = 0$.
  From here we clearly have $\lambda = 0$.
  
  Conversely, assume that $\sigma(L) = \{0\}$.
\end{solution}

\end{document}
