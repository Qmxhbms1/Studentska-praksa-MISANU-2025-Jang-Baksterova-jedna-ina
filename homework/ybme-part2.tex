\documentclass{article}

% ===================================================================
% PREAMBLE FOR MATHEMATICS EXERCISE NOTEBOOK
% ===================================================================
% This file contains all the packages and custom commands for the document.
% Keeping it separate from main.tex makes the project cleaner.


% -------------------------------------------------------------------
% DOCUMENT & ENCODING
% -------------------------------------------------------------------
\usepackage[utf8]{inputenc} % Allows you to type characters like á, ö, etc. directly
\usepackage[T1]{fontenc}    % Specifies font encoding, improves font rendering and hyphenation


% -------------------------------------------------------------------
% PAGE LAYOUT & GEOMETRY
% -------------------------------------------------------------------
\usepackage{geometry}
\geometry{
  a4paper,         % Or letterpaper, etc.
  total={170mm,257mm},
  left=20mm,
  top=20mm,
}


% -------------------------------------------------------------------
% CORE MATH PACKAGES (AMS - American Mathematical Society)
% -------------------------------------------------------------------
\usepackage{amsmath}    % The premier package for typesetting math equations
\usepackage{amssymb}    % Provides lots of extra math symbols (like \mathbb)
\usepackage{amsthm}     % Comprehensive theorem-like environments
\usepackage{mathtools}  % An extension of amsmath, provides more tools and fixes


% -------------------------------------------------------------------
% UTILITY & FORMATTING PACKAGES
% -------------------------------------------------------------------
\usepackage{graphicx}      % For including images (\includegraphics)
\usepackage{xcolor}        % For defining and using colors
\usepackage{booktabs}      % For creating beautiful, professional-looking tables (\toprule, \midrule, \bottomrule)
\usepackage{siunitx}       % For typesetting numbers and units consistently
\usepackage{enumitem}      % Provides more control over list environments (itemize, enumerate)


% -------------------------------------------------------------------
% HYPERLINKS & CROSS-REFERENCING
% -------------------------------------------------------------------
\usepackage{hyperref}
\hypersetup{
    colorlinks=true,       % false: boxed links; true: colored links
    linkcolor=teal,        % color of internal links (e.g., sections)
    citecolor=green,       % color of links to bibliography
    filecolor=magenta,     % color of file links
    urlcolor=blue          % color of external links
}


% -------------------------------------------------------------------
% THEOREM & DEFINITION ENVIRONMENTS
% -------------------------------------------------------------------
% This section sets up consistent numbering and styling for problems,
% definitions, theorems, etc.

\theoremstyle{definition} % Use a style that is less flashy than the default "plain" style
\newtheorem{problem}{Problem} % Number problems as Problem X.Y (where X is the section number)
\newtheorem{theorem}[problem]{Theorem}     % Share the same counter as 'problem'
\newtheorem{lemma}[problem]{Lemma}
\newtheorem{corollary}[problem]{Corollary}
\newtheorem{definition}[problem]{Definition}
\newtheorem{example}[problem]{Example}
\newtheorem{remark}[problem]{Remark}


% -------------------------------------------------------------------
% CUSTOM SOLUTION ENVIRONMENT
% -------------------------------------------------------------------
% We create a new "solution" environment that looks clean and
% automatically adds a QED symbol at the end.

\newenvironment{solution}
  {
   % Begin environment
   \renewcommand\qedsymbol{$\blacksquare$} % Change the default QED symbol to a black square
   \begin{proof}[Solution] % Base it on the 'proof' environment but change the title
  }
  {
   % End environment
   \end{proof}
  }


% -------------------------------------------------------------------
% CUSTOM MATH COMMANDS (MACROS)
% -------------------------------------------------------------------
% Define shortcuts for commonly used mathematical notation to save
% time and ensure consistency.

% Sets of numbers
\newcommand{\R}{\mathbb{R}} % Real numbers
\newcommand{\C}{\mathbb{C}} % Complex numbers
\newcommand{\N}{\mathbb{N}} % Natural numbers
\newcommand{\Z}{\mathbb{Z}} % Integers
\newcommand{\Q}{\mathbb{Q}} % Rational numbers

% Calculus operators
\newcommand{\dd}{\, \mathrm{d}} % For integrals, e.g., \int f(x)\dd x
\newcommand{\pdv}[2]{\frac{\partial #1}{\partial #2}} % Partial derivative
\newcommand{\dv}[2]{\frac{\mathrm{d} #1}{\mathrm{d} #2}} % Full derivative

% Linear Algebra
\DeclareMathOperator{\Tr}{Tr} % Trace of a matrix
\newcommand{\T}{\mathsf{T}}   % Transpose, e.g., A^\T

% Probability & Statistics
\DeclareMathOperator{\E}{\mathbb{E}} % Expectation
\DeclareMathOperator{\Var}{Var} % Variance
\DeclareMathOperator{\Cov}{Cov} % Covariance


% ===================================================================
% END OF PREAMBLE
% ===================================================================


\title{Yang-Baxter-like matrix equation}
\author{Mihailo Đurić}
\date{\today}

\begin{document}

\maketitle
\newpage

\begin{problem}
  Prove that nonzero orthogonal vectors are mutually linearly independent.
\end{problem}

\begin{solution}
  Let $u$ and $v$ be nonzero orthogonal vectors, and for a contradiction assume they are linearly dependent.
  In other words, we know that $u = t v$ for some scalar $t$.
  Thus we have the following
  \[0 = \langle u, v \rangle = \langle t v, v \rangle = t \langle v, v \rangle.\]
  Thus we see that either $t = 0$, contradicting that $u$ and $v$ are non-zero, or $\langle v, v \rangle = 0$.
  However, we know that $\langle v, v \rangle = 0$ if and only if $v = 0$, so again have a contradiction.
\end{solution}

\begin{problem}
  Prove that
  \[L^* = \overline{L^t}.\]
\end{problem}

\begin{solution}
  First we look at
  \[\langle Lu, v \rangle = \sum_{k=1}^{n} (Lu)_k \overline{v_k} = \sum_{k=1}^{n} \sum_{i=1}^{n} L_{ki} u_i \overline{v_k}.\]
  On the other hand, we have
  \[\langle u, \overline{L^t} v \rangle = \sum_{k=1}^{n} u_k (L^t \overline{v})_k = \sum_{k=1}^{n} u_k (\sum_{i=1}^{n} L_{ki}^t \overline{v_i}) = \sum_{k=1}^{n} \sum_{i=1}^{n} u_k L_{ik} \overline{v_i}.\]
  Notice that these two sums are exactly equal (just swap $k$ and $i$ as variables in the second equation).

  To show that this matrix is unique, suppose that there were two matrice $L_1^*$ and $L_2^*$ which both satisfied $\langle L u, v \rangle = \langle u, L_i^* v \rangle$.
  Then we clearly see that $\langle u, L_1^* v \rangle = \langle u, L_2^* v \rangle$.
  By the linearity of the inner product we then know that $\langle u, (L_1^* - L_2^*) v \rangle = 0$.
  We thus know that either $u = 0$ or $(L_1^* - L_2^*) v = 0$.
  Since this equation holds for arbitrary $u$ and $v$, we have that $L_1^* - L_2^* = 0$.
  Finally, we conclude that $L_1^* = L_2^*$.
  Thus $L^* = \overline{L^t}$.
\end{solution}

\begin{problem}
  Prove that
  \[\sigma (L^*) = \overline{\sigma{L}}.\]
\end{problem}

\begin{solution}
  We have
  \[\sigma(L^*) = \sigma(\overline{L^t}) = \sigma(\overline{L}) = \overline{\sigma(L)}.\]
  The first equality holds by the previous exercise and the second holds from the fact that transposing a matrix doesn't change the eigenvalues.
  The finally equality stems from the fact that taking the complex conjugate of a matrix is the same as a change of basis.
  Since the eigenvalues do not depend on the basis, it is clear that the equality holds.
\end{solution}

\begin{problem}
  Prove that the spectrum of a unitary matrix $U$ is contained on the complex unit sphere:
  \[\sigma(U) \subset \{z \in \C : |z| = 1\}.\]
\end{problem}

\begin{solution}
  Let $\lambda$ be any eigenvalue of $U$, so $U x = \lambda x$.
  By the properties of the conjugate transpose (shown in the next exercise) we know that $(U x)^* = x^* U^*$ and $(U x)^* = (\lambda x)^* = \overline{\lambda} x^*$.
  Multiplying our equation we have
  \[\begin{aligned}
    x^* U^* U x &= \overline{\lambda} x^* \lambda x
    x^* x &= (\overline{\lambda}) \lambda x^* x
    ||x||^2 &= |\lambda|^2 ||x||^2
    1 &= |\lambda|^2
    1 &= |\lambda|
  \end{aligned}\]
\end{solution}

\begin{problem}
  Prove that a product of two unitary matrices is a unitary matrix.
\end{problem}

\begin{solution}
  Let $U$ and $V$ be two unitary matrices with inverses $U^*$ and $V^*$.
  We know that the inverse of $UV$ is precisely $V^* U^*$.
  Now we need to show that $V^* U^* = (UV)^*$.
  From the propetries of transposed matrices we know that $(AB)^t = A^t B^t$, hence $\overline{UV^t} = \overline{V^t U^t}$.
  For some matrices $A$ and $B$ we can see that the general element of $\overline{AB}$ is
  \[\overline{(AB)_ij} = \overline{\sum_{k=1}^{n} A_i B_j} = \sum_{k=1}^{n} \bar{A_i} \bar{B_j}.\]
  Hence $\overline{AB} = \bar{A} \bar{B}$.
  Thus we have $(UV)^* = \overline{UV^t} = \overline{V^t U^t} = \bar{V^t} \bar{U^t} = V^* U^*$.
\end{solution}

\begin{problem}
  Prove the following theorem

  Let $A$ be a square matrix, and let $f$ be a well-defined function in the neighborhood of $\sigma(A)$,
  such that $f$ has no zeros in $\sigma(A)$.
  If $X_0$ is a non-commuting solution to YBME, the so is the matrix $f(A) X_0 f(A)^{-1}$.
\end{problem}

\begin{solution}
  Let $X_0$ be a non-commuting solution to the YBME, i.e., $A X_0 A = X_0 A X_0$.
  Since $f$ has no zeros in $\sigma(A)$ it follows that $f(A)$ is invertible.
  Also, note that $A$ and $f(A)$ commute, hence $A$ and $f^{-1}(A)$ also commute.
  Then
  \[\begin{aligned}
    0 &= f(A) A X_0 A f^{-1}(A) - f(A) A X_0 A f^{-1}(A)\\ 
    &= f(A) X_0 A X_0 f^{-1} - f(A) A X_0 A f^{-1}(A)\\
    &= f(A) X_0 f^{-1}(A) f(A) A X_0 A f^{-1}(A) - A f(A) X_0 f^{-1}(A) A\\
    &= f(A) X_0 f^{-1}(A) A f(A) X_0 f^{-1}(A) - A f(A) X_0 f^{-1} A
  \end{aligned}\]
  Hence we have $f(A) X_0 f^{-1}(A) A f(A) X_0 f^{-1}(A) = A f(A) X_0 f^{-1} A$, and $f(A) X_0 f^{-1}(A)$ is a solution to the YBME.
\end{solution}

\begin{problem}
  Let $A$ be an arbitrary square matrix, and suppose that $X_0$ is one of its non-commuting solutions.
  Show that for every scalar $t$, it follows that $e^{At} X_0 e^{-At}$ is also a solution to YBME.
\end{problem}

\begin{solution}
  It is clear that $e^{At}$ is a well-defined function for any matrix $A$ and any $t$.
  We also know that $e^x$ is never equal to zero for any value of $x$.
  Finally we know that $(e^{At})^{-1} = e^{-At}$.
  Thus by applying the previous problem we can immediately see that if $X_0$ is a solution to the YBME then $e^{At} X_0 e^{-At}$ must also be a solution.
\end{solution}

\end{document}
