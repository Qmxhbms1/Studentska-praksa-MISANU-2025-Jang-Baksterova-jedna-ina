\documentclass{article}

% ===================================================================
% PREAMBLE FOR MATHEMATICS EXERCISE NOTEBOOK
% ===================================================================
% This file contains all the packages and custom commands for the document.
% Keeping it separate from main.tex makes the project cleaner.


% -------------------------------------------------------------------
% DOCUMENT & ENCODING
% -------------------------------------------------------------------
\usepackage[utf8]{inputenc} % Allows you to type characters like á, ö, etc. directly
\usepackage[T1]{fontenc}    % Specifies font encoding, improves font rendering and hyphenation


% -------------------------------------------------------------------
% PAGE LAYOUT & GEOMETRY
% -------------------------------------------------------------------
\usepackage{geometry}
\geometry{
  a4paper,         % Or letterpaper, etc.
  total={170mm,257mm},
  left=20mm,
  top=20mm,
}


% -------------------------------------------------------------------
% CORE MATH PACKAGES (AMS - American Mathematical Society)
% -------------------------------------------------------------------
\usepackage{amsmath}    % The premier package for typesetting math equations
\usepackage{amssymb}    % Provides lots of extra math symbols (like \mathbb)
\usepackage{amsthm}     % Comprehensive theorem-like environments
\usepackage{mathtools}  % An extension of amsmath, provides more tools and fixes


% -------------------------------------------------------------------
% UTILITY & FORMATTING PACKAGES
% -------------------------------------------------------------------
\usepackage{graphicx}      % For including images (\includegraphics)
\usepackage{xcolor}        % For defining and using colors
\usepackage{booktabs}      % For creating beautiful, professional-looking tables (\toprule, \midrule, \bottomrule)
\usepackage{siunitx}       % For typesetting numbers and units consistently
\usepackage{enumitem}      % Provides more control over list environments (itemize, enumerate)


% -------------------------------------------------------------------
% HYPERLINKS & CROSS-REFERENCING
% -------------------------------------------------------------------
\usepackage{hyperref}
\hypersetup{
    colorlinks=true,       % false: boxed links; true: colored links
    linkcolor=teal,        % color of internal links (e.g., sections)
    citecolor=green,       % color of links to bibliography
    filecolor=magenta,     % color of file links
    urlcolor=blue          % color of external links
}


% -------------------------------------------------------------------
% THEOREM & DEFINITION ENVIRONMENTS
% -------------------------------------------------------------------
% This section sets up consistent numbering and styling for problems,
% definitions, theorems, etc.

\theoremstyle{definition} % Use a style that is less flashy than the default "plain" style
\newtheorem{problem}{Problem} % Number problems as Problem X.Y (where X is the section number)
\newtheorem{theorem}[problem]{Theorem}     % Share the same counter as 'problem'
\newtheorem{lemma}[problem]{Lemma}
\newtheorem{corollary}[problem]{Corollary}
\newtheorem{definition}[problem]{Definition}
\newtheorem{example}[problem]{Example}
\newtheorem{remark}[problem]{Remark}


% -------------------------------------------------------------------
% CUSTOM SOLUTION ENVIRONMENT
% -------------------------------------------------------------------
% We create a new "solution" environment that looks clean and
% automatically adds a QED symbol at the end.

\newenvironment{solution}
  {
   % Begin environment
   \renewcommand\qedsymbol{$\blacksquare$} % Change the default QED symbol to a black square
   \begin{proof}[Solution] % Base it on the 'proof' environment but change the title
  }
  {
   % End environment
   \end{proof}
  }


% -------------------------------------------------------------------
% CUSTOM MATH COMMANDS (MACROS)
% -------------------------------------------------------------------
% Define shortcuts for commonly used mathematical notation to save
% time and ensure consistency.

% Sets of numbers
\newcommand{\R}{\mathbb{R}} % Real numbers
\newcommand{\C}{\mathbb{C}} % Complex numbers
\newcommand{\N}{\mathbb{N}} % Natural numbers
\newcommand{\Z}{\mathbb{Z}} % Integers
\newcommand{\Q}{\mathbb{Q}} % Rational numbers

% Calculus operators
\newcommand{\dd}{\, \mathrm{d}} % For integrals, e.g., \int f(x)\dd x
\newcommand{\pdv}[2]{\frac{\partial #1}{\partial #2}} % Partial derivative
\newcommand{\dv}[2]{\frac{\mathrm{d} #1}{\mathrm{d} #2}} % Full derivative

% Linear Algebra
\DeclareMathOperator{\Tr}{Tr} % Trace of a matrix
\newcommand{\T}{\mathsf{T}}   % Transpose, e.g., A^\T

% Probability & Statistics
\DeclareMathOperator{\E}{\mathbb{E}} % Expectation
\DeclareMathOperator{\Var}{Var} % Variance
\DeclareMathOperator{\Cov}{Cov} % Covariance


% ===================================================================
% END OF PREAMBLE
% ===================================================================


\title{Linear Operators}
\author{Mihailo Đurić}
\date{\today}

\begin{document}

\maketitle
\newpage

\begin{problem}
  Let $A = \begin{bmatrix} \cos(\frac{2 \pi}{n}) & -\sin(\frac{2 \pi}{n})\\ \sin(\frac{2 \pi}{n}) & \cos(\frac{2 \pi}{n}) \end{bmatrix}, n \in N, B = \begin{bmatrix} 0 & 1\\ \pi & \sqrt{2} \end{bmatrix}$.
  
  \begin{enumerate}[label=(\alph*)]
    \item Show that $AB \neq BA$.
    \item Show that $A^n = I$.
    \item Show that $A^nB = BA^n$.
  \end{enumerate}

  What does this demonstrate?
\end{problem}

\begin{solution}
  \begin{enumerate}[label=(\alph*)]
    \item First we have
        \[AB = \begin{bmatrix} \cos(\frac{2 \pi}{n}) & -\sin(\frac{2 \pi}{n})\\ \sin(\frac{2 \pi}{n}) & \cos(\frac{2 \pi}{n}) \end{bmatrix} \cdot \begin{bmatrix} 0 & 1\\ \pi & \sqrt{2} \end{bmatrix} = \begin{bmatrix} \sin(\frac{2 \pi}{n}) & \cos(\frac{2 \pi}{n})\\ \pi \cos(\frac{2 \pi}{n}) + \sqrt{2} \sin(\frac{2 \pi}{n}) & -\pi \sin(\frac{2 \pi}{n}) + \sqrt{2} \cos(\frac{2 \pi}{n}) \end{bmatrix}.\]
        On the other hand we have
        \[BA = \begin{bmatrix} 0 & 1\\ \pi & \sqrt{2} \end{bmatrix} \cdot \begin{bmatrix} \cos(\frac{2 \pi}{n}) & -\sin(\frac{2 \pi}{n})\\ \sin(\frac{2 \pi}{n}) & \cos(\frac{2 \pi}{n}) \end{bmatrix} = \begin{bmatrix} - \pi \sin(\frac{2 \pi}{n}) & \cos(\frac{2 \pi}{n}) - \sqrt{2} \sin(\frac{2 \pi}{n})\\ \pi \cos(\frac{2 \pi}{n}) & \sin(\frac{2 \pi}{n}) + \sqrt{2} \cos(\frac{2 \pi}{n}) \end{bmatrix}.\]
        We can clearly see that these matrices are not equal for every $n \in N$ as clearly $\sin(2 \pi) \neq \pi \sin(2 \pi)$ for $n = 1$.
    \item We will show by induction that $A^k = \begin{bmatrix} \cos(k \cdot \frac{2 \pi}{n}) & -\sin(k \cdot \frac{2 \pi}{n})\\ \sin(k \cdot \frac{2 \pi}{n}) & \cos(k \cdot \frac{2 \pi}{n}) \end{bmatrix}$ for some $k \in \N$.
        The base case for $k = 1$ holds trivially.
        Assume now that $A^{k - 1} = \begin{bmatrix} \cos((k - 1) \cdot \frac{2 \pi}{n}) & -\sin((k - 1) \cdot \frac{2 \pi}{n})\\ \sin((k - 1) \cdot \frac{2 \pi}{n}) & \cos((k - 1) \cdot \frac{2 \pi}{n}) \end{bmatrix}$.
        Now consider
        \[\begin{aligned}
          A^k = AA^{k - 1} &= \begin{bmatrix} \cos((k - 1) \cdot \frac{2 \pi}{n}) & -\sin((k - 1) \cdot \frac{2 \pi}{n})\\ \sin((k - 1) \cdot \frac{2 \pi}{n}) & \cos((k - 1) \cdot \frac{2 \pi}{n}) \end{bmatrix} \cdot \begin{bmatrix} \cos(\frac{2 \pi}{n}) & -\sin(\frac{2 \pi}{n})\\ \sin(\frac{2 \pi}{n}) & \cos(\frac{2 \pi}{n}) \end{bmatrix}\\
          &= \begin{bmatrix} \cos(\frac{2 \pi}{n}) \cos((k - 1) \frac{2 \pi}{n}) - \sin(\frac{2 \pi}{n}) \sin((k - 1) \frac{2 \pi}{n}) & - \cos(\frac{2 \pi}{n}) \sin((k - 1) \frac{2 \pi}{n}) + \sin(\frac{2 \pi}{n}) \cos((k - 1) \frac{2 \pi}{n})\\ \sin(\frac{2 \pi}{n}) \cos((k - 1) \frac{2 \pi}{n}) + \cos(\frac{2 \pi}{n}) \sin((k - 1) \frac{2 \pi}{n}) & - \sin(\frac{2 \pi}{n}) \sin((k - 1) \frac{2 \pi}{n}) + \cos(\frac{2 \pi}{n}) \cos((k - 1) \frac{2 \pi}{n}) \end{bmatrix}\\
          &= \begin{bmatrix} \cos(k \cdot \frac{2 \pi}{n}) & -\sin(k \cdot \frac{2 \pi}{n})\\ \sin(k \cdot \frac{2 \pi}{n}) & \cos(k \cdot \frac{2 \pi}{n}) \end{bmatrix}.
        \end{aligned}\]
        Hence, if we put $k = n$ we will have
        \[A^n = \begin{bmatrix} \cos(n \cdot \frac{2 \pi}{n}) & -\sin(n \cdot \frac{2 \pi}{n})\\ \sin(n \cdot \frac{2 \pi}{n}) & \cos(n \cdot \frac{2 \pi}{n}) \end{bmatrix} = \begin{bmatrix} \cos(2 \pi) & - \sin(2 \pi)\\ \sin(2 \pi) & \cos(2 \pi) \end{bmatrix} = \begin{bmatrix} 1 & 0\\ 0 & 1 \end{bmatrix} = I.\]
    \item By the definition of $I$ and (b) we have the following
      \[A^nB = IB = BI = BA^n.\]
  \end{enumerate}

  We have demonstrated that $AB = BA$ and $A^nB = BA^n$ are not equivalent.
  While the first implies the second, the converse does not hold as we have shown.
\end{solution}

\begin{problem}
  If some $M \in L(V)$ commutes with $L$, then $(\forall n \in \N_0)$ $M$ commutes with $L^n$.
  Then, for every choice $(\alpha_0, \alpha_1, \ldots, \alpha_n \in \mathbb{K})$, the operator polynomial
  \[p_n(L) := \alpha_0 I + \alpha_1 L + \alpha_2 L^2 + \ldots + \alpha_n L^n = \sum_{k = 0}^{n} \alpha_k L^k\]
  commutes with $M$.
\end{problem}

\begin{solution}
  Let us consider $M p_n(L) = M \sum_{k = 0}^{n} \alpha_k L^k$.
  Because of the linearity of $M$ we have
  \[M \sum_{k = 0}^{n} \alpha_k L^k = M \cdot \alpha_0 I + M \cdot \alpha_1 L + \ldots + M \cdot \alpha_n L^n.\]
  Since scalar multiplication is associative we know
  \[M \cdot \alpha_0 I + M \cdot \alpha_1 L + \ldots + M \cdot \alpha_n L^n = \alpha_0 M I + \alpha_1 M L + \ldots + \alpha_n M L^n.\]
  As we know that $M$ commutes with any power of $n$ we have
  \[\alpha_0 M I + \alpha_1 M L + \ldots + \alpha_n M L^n = \alpha_0 I M + \alpha_1 L M + \ldots + \alpha_n L^n M.\]
  Finally, from the linearity of $M$ again we have
  \[\alpha_0 I M + \alpha_1 L M + \ldots + \alpha_n L^n M = (\sum_{k = 0}^{n} \alpha_k L^K) \cdot M = p_n(L) M.\]
\end{solution}

\begin{problem}
  If $\lambda_1, \ldots, \lambda_n$ are different eigenvalues for $L$ then the corresponding eigenspaces (nullspaces of $L - \lambda_i I$) are linearly independent.
\end{problem}

\begin{solution}
  Let $u_1, \ldots, u_n$ be eigenvectors such that $Lu_i = \lambda_i u_i$.
  Assume the contrary, let this set of vectors be linearly dependent.
  Let $u_k$ be the least $k \in N$ such that $u_1, \ldots, u_k$ is linearly dependent.
  Then we have
  \[u_k = - \sum_{i = 0}^{k - 1} \frac{\alpha_i}{\alpha_k} u_i, \text{ such that $\alpha_k \neq 0$.}\]
  First if we apply $L$ to the left side we get $Lu_k = \lambda_k u_k$.
  From this we get $Lu_k = \lambda_k (- \sum_{i = 0}^{k - 1} \frac{\alpha_i}{\alpha_k} u_i) = - \sum_{i = 0}^{k - 1} \frac{\alpha_i}{\alpha_k} \lambda_k u_i$.
  Now applying it to the right side of the equality we get
  \[L(- \sum_{i = 0}^{k - 1} \frac{\alpha_i}{\alpha_k} u_i) = - \sum_{i = 0}^{k - 1} \frac{\alpha_i}{\alpha_k} \lambda_i u_i.\]
  Combining this and out above equation we get
  \[- \sum_{i = 0}^{k - 1} \frac{\alpha_i}{\alpha_k} \lambda_k u_i = - \sum_{i = 0}^{k - 1} \frac{\alpha_i}{\alpha_k} \lambda_i u_i.\]
  Simplifying this a little bit we get
  \[\sum_{i = 0}^{k - 1} \alpha_i (\lambda_k - \lambda_i) u_i = 0.\]
  Since the vectors $u_1, \ldots, u_{k - 1}$ are linearly independent by the choice of $k$ we know that $\alpha_i (\lambda_k - \lambda_i) = 0$ for every $i$.
  From our assumption that $\lambda_1, \ldots, \lambda_n$ are different eigenvalues we know that $\lambda_k \neq \lambda_i$ for any $i$.
  Thus we need $\alpha_i = 0$ for every $i$ to hold.
  However, in that case notice that $u_k = 0$, contradicting that eigenvectors are non-zero vectors.
  From here we see that our set of vectors $u_i$ must be linearly independent.
\end{solution}

\end{document}
