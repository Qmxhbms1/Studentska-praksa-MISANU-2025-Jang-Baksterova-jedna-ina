% ===================================================================
% PREAMBLE FOR MATHEMATICS EXERCISE NOTEBOOK
% ===================================================================
% This file contains all the packages and custom commands for the document.
% Keeping it separate from main.tex makes the project cleaner.


% -------------------------------------------------------------------
% DOCUMENT & ENCODING
% -------------------------------------------------------------------
\usepackage[utf8]{inputenc} % Allows you to type characters like á, ö, etc. directly
\usepackage[T1]{fontenc}    % Specifies font encoding, improves font rendering and hyphenation


% -------------------------------------------------------------------
% PAGE LAYOUT & GEOMETRY
% -------------------------------------------------------------------
\usepackage{geometry}
\geometry{
  a4paper,         % Or letterpaper, etc.
  total={170mm,257mm},
  left=20mm,
  top=20mm,
}


% -------------------------------------------------------------------
% CORE MATH PACKAGES (AMS - American Mathematical Society)
% -------------------------------------------------------------------
\usepackage{amsmath}    % The premier package for typesetting math equations
\usepackage{amssymb}    % Provides lots of extra math symbols (like \mathbb)
\usepackage{amsthm}     % Comprehensive theorem-like environments
\usepackage{mathtools}  % An extension of amsmath, provides more tools and fixes


% -------------------------------------------------------------------
% UTILITY & FORMATTING PACKAGES
% -------------------------------------------------------------------
\usepackage{graphicx}      % For including images (\includegraphics)
\usepackage{xcolor}        % For defining and using colors
\usepackage{booktabs}      % For creating beautiful, professional-looking tables (\toprule, \midrule, \bottomrule)
\usepackage{siunitx}       % For typesetting numbers and units consistently
\usepackage{enumitem}      % Provides more control over list environments (itemize, enumerate)


% -------------------------------------------------------------------
% HYPERLINKS & CROSS-REFERENCING
% -------------------------------------------------------------------
\usepackage{hyperref}
\hypersetup{
    colorlinks=true,       % false: boxed links; true: colored links
    linkcolor=teal,        % color of internal links (e.g., sections)
    citecolor=green,       % color of links to bibliography
    filecolor=magenta,     % color of file links
    urlcolor=blue          % color of external links
}


% -------------------------------------------------------------------
% THEOREM & DEFINITION ENVIRONMENTS
% -------------------------------------------------------------------
% This section sets up consistent numbering and styling for problems,
% definitions, theorems, etc.

\theoremstyle{definition} % Use a style that is less flashy than the default "plain" style
\newtheorem{problem}{Problem}[section] % Number problems as Problem X.Y (where X is the section number)
\newtheorem{theorem}[problem]{Theorem}     % Share the same counter as 'problem'
\newtheorem{lemma}[problem]{Lemma}
\newtheorem{corollary}[problem]{Corollary}
\newtheorem{definition}[problem]{Definition}
\newtheorem{example}[problem]{Example}
\newtheorem{remark}[problem]{Remark}


% -------------------------------------------------------------------
% CUSTOM SOLUTION ENVIRONMENT
% -------------------------------------------------------------------
% We create a new "solution" environment that looks clean and
% automatically adds a QED symbol at the end.

\newenvironment{solution}
  {
   % Begin environment
   \renewcommand\qedsymbol{$\blacksquare$} % Change the default QED symbol to a black square
   \begin{proof}[Solution] % Base it on the 'proof' environment but change the title
  }
  {
   % End environment
   \end{proof}
  }


% -------------------------------------------------------------------
% CUSTOM MATH COMMANDS (MACROS)
% -------------------------------------------------------------------
% Define shortcuts for commonly used mathematical notation to save
% time and ensure consistency.

% Sets of numbers
\newcommand{\R}{\mathbb{R}} % Real numbers
\newcommand{\C}{\mathbb{C}} % Complex numbers
\newcommand{\N}{\mathbb{N}} % Natural numbers
\newcommand{\Z}{\mathbb{Z}} % Integers
\newcommand{\Q}{\mathbb{Q}} % Rational numbers

% Calculus operators
\newcommand{\dd}{\, \mathrm{d}} % For integrals, e.g., \int f(x)\dd x
\newcommand{\pdv}[2]{\frac{\partial #1}{\partial #2}} % Partial derivative
\newcommand{\dv}[2]{\frac{\mathrm{d} #1}{\mathrm{d} #2}} % Full derivative

% Linear Algebra
\DeclareMathOperator{\Tr}{Tr} % Trace of a matrix
\newcommand{\T}{\mathsf{T}}   % Transpose, e.g., A^\T

% Probability & Statistics
\DeclareMathOperator{\E}{\mathbb{E}} % Expectation
\DeclareMathOperator{\Var}{Var} % Variance
\DeclareMathOperator{\Cov}{Cov} % Covariance


% ===================================================================
% END OF PREAMBLE
% ===================================================================
