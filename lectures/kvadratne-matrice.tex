\documentclass{article}

% ===================================================================
% PREAMBLE FOR MATHEMATICS EXERCISE NOTEBOOK
% ===================================================================
% This file contains all the packages and custom commands for the document.
% Keeping it separate from main.tex makes the project cleaner.


% -------------------------------------------------------------------
% DOCUMENT & ENCODING
% -------------------------------------------------------------------
\usepackage[utf8]{inputenc} % Allows you to type characters like á, ö, etc. directly
\usepackage[T1]{fontenc}    % Specifies font encoding, improves font rendering and hyphenation


% -------------------------------------------------------------------
% PAGE LAYOUT & GEOMETRY
% -------------------------------------------------------------------
\usepackage{geometry}
\geometry{
  a4paper,         % Or letterpaper, etc.
  total={170mm,257mm},
  left=20mm,
  top=20mm,
}


% -------------------------------------------------------------------
% CORE MATH PACKAGES (AMS - American Mathematical Society)
% -------------------------------------------------------------------
\usepackage{amsmath}    % The premier package for typesetting math equations
\usepackage{amssymb}    % Provides lots of extra math symbols (like \mathbb)
\usepackage{amsthm}     % Comprehensive theorem-like environments
\usepackage{mathtools}  % An extension of amsmath, provides more tools and fixes


% -------------------------------------------------------------------
% UTILITY & FORMATTING PACKAGES
% -------------------------------------------------------------------
\usepackage{graphicx}      % For including images (\includegraphics)
\usepackage{xcolor}        % For defining and using colors
\usepackage{booktabs}      % For creating beautiful, professional-looking tables (\toprule, \midrule, \bottomrule)
\usepackage{siunitx}       % For typesetting numbers and units consistently
\usepackage{enumitem}      % Provides more control over list environments (itemize, enumerate)


% -------------------------------------------------------------------
% HYPERLINKS & CROSS-REFERENCING
% -------------------------------------------------------------------
\usepackage{hyperref}
\hypersetup{
    colorlinks=true,       % false: boxed links; true: colored links
    linkcolor=teal,        % color of internal links (e.g., sections)
    citecolor=green,       % color of links to bibliography
    filecolor=magenta,     % color of file links
    urlcolor=blue          % color of external links
}

\renewcommand{\contentsname}{Sadžaj}

% -------------------------------------------------------------------
% THEOREM & DEFINITION ENVIRONMENTS
% -------------------------------------------------------------------
% This section sets up consistent numbering and styling for problems,
% definitions, theorems, etc.

\theoremstyle{definition} % Use a style that is less flashy than the default "plain" style
\newtheorem{definition}{Definicija}[section] % Number problems as Problem X.Y (where X is the section number)
\newtheorem{example}[definition]{Primer}

\theoremstyle{plain}
\newtheorem{theorem}[definition]{Teorema}     % Share the same counter as 'problem'
\newtheorem{lemma}[definition]{Lema}
\newtheorem{corollary}[definition]{Posledica}

\theoremstyle{remark}
\newtheorem{remark}[definition]{Primetiti}

\renewcommand{\proofname}{Dokaz}
% -------------------------------------------------------------------
% CUSTOM MATH COMMANDS (MACROS)
% -------------------------------------------------------------------
% Define shortcuts for commonly used mathematical notation to save
% time and ensure consistency.

% Sets of numbers
\newcommand{\R}{\mathbb{R}} % Real numbers
\newcommand{\C}{\mathbb{C}} % Complex numbers
\newcommand{\N}{\mathbb{N}} % Natural numbers
\newcommand{\Z}{\mathbb{Z}} % Integers
\newcommand{\Q}{\mathbb{Q}} % Rational numbers

% Calculus operators
\newcommand{\dd}{\, \mathrm{d}} % For integrals, e.g., \int f(x)\dd x
\newcommand{\pdv}[2]{\frac{\partial #1}{\partial #2}} % Partial derivative
\newcommand{\dv}[2]{\frac{\mathrm{d} #1}{\mathrm{d} #2}} % Full derivative

% Linear Algebra
\DeclareMathOperator{\Tr}{Tr} % Trace of a matrix
\newcommand{\T}{\mathsf{T}}   % Transpose, e.g., A^\T

% Probability & Statistics
\DeclareMathOperator{\E}{\mathbb{E}} % Expectation
\DeclareMathOperator{\Var}{Var} % Variance
\DeclareMathOperator{\Cov}{Cov} % Covariance


% ===================================================================
% END OF PREAMBLE
% ===================================================================


\title{Kvadratne matrice kao linearna preslikavanja na vektorskom prostoru}
\author{Mihailo Đurić}
\date{\today}

\begin{document}

\maketitle
\tableofcontents
\newpage

\section{Sistemi linearno zavisnih i nezavisnih vektora}

\begin{definition}[Vektorski prostor]
  \textit{Vektorski prostor nad poljem $\mathbb{F}$ (elementi $\mathbb{F}$ zovu se skalari)} je skup $\mathcal{V}$ (elementi $\mathcal{V}$ zovu se vektori) sa binarnom operacijom vektorskog sabiranja i binarnom funkcijom skalarnog množenja.

  Vektorsko sabiranje je binarna operacija $+ : \mathcal{V} \times \mathcal{V} \to \mathcal{V}$ koja svakom paru vektora $x, y \in \mathcal{V}$ dodeljuje vektor $x + y \in \mathcal{V}$ tako da
  \begin{enumerate}
    \item Sabiranje je komutativno, $x + y = y + x$;
    \item Sabiranje je asocijativno, $x + (y + z) = (x + y) + z$;
    \item Postoji jedinstven vektor $0 \in \mathcal{V}$ tako da $x + 0 = x$ za svaki vektor $x$;
    \item Za svaki vektor $x \in \mathcal{V}$ postoji odgovarajući vektor $-x$ tako da $x + (-x) = 0$.
  \end{enumerate}
  
  Skalarno množenje je binarna funkcija $\cdot : \mathbb{F} \times \mathcal{V} \to \mathcal{V}$ koja svakom paru $\alpha \in \mathbb{F}$ i $x \in \mathcal{V}$ dodeljuje vektor $\alpha x \in \mathcal{V}$ tako da
  \begin{enumerate}
    \item Skalarno množenje je asocijativno, $\alpha (\beta x) = (\alpha \beta) x$;
    \item $1 \cdot x = x$ za svaki vektor $x$;
    \item Skalarno množenje je distributivno prema vektorskom sabiranju, $\alpha (x + y) = \alpha x + \alpha y$;
    \item Skalarno množenje je distributivno prema sabiranju skalara, $(\alpha + \beta) x = \alpha x + \beta x$.
  \end{enumerate}
\end{definition}

\begin{definition}[Vektorski potprostor]
  \textit{Potprostor} $\mathcal{W}$ je podskup vektorskog prostora $\mathcal{V}$ koji je i sam vektorski prostor nad poljem $\mathbb{F}$ sa istim operacijama kao i $\mathcal{V}$
\end{definition}

\begin{theorem}
  Neprazan skup $\mathcal{W} \subset \mathcal{V}$ je potprostor vektorskog prostora $\mathcal{V}$ nad poljem $\mathbb{F}$ ako i samo ako za svaki $w_1, w_2 \in \mathcal{W}$ i $\alpha, \beta \in \mathbb{F}$, sledi da je $\alpha w_1 + \beta w_2 \in \mathcal{W}$.
\end{theorem}

\begin{proof}
  Neka je $\mathcal{W}$ neprazan podskup $\mathcal{V}$ tako da važi da za svaki $w_1, w_2 \in \mathcal{W}$ i $\alpha, \beta \in \mathbb{F}$, sledi da je $\alpha w_1 + \beta w_2 \in \mathcal{W}$.
  Za bilo koji vektor $x \in \mathcal{W}$ važi $(-1) x + x = 0$, odakle $0 \in \mathcal{W}$.
  Ako je $x \in \mathcal{V}$ i $\alpha \in \mathbb{F}$ onda znamo da je $\alpha x + 0 = \alpha x$, dakle $\alpha x \in \mathcal{W}$.
  Specifično, $(-1) x = -x$ je u $\mathcal{W}$.
  Konačno, ako su $x, y \in \mathcal{W}$ onda je $(1) x + (1) y = x + y$ isto u $\mathcal{W}$.
  Asocijativnosti i komutativnost sabiranja se nasledjuje iz $\mathcal{V}$, kao i asocijativnost i distributivnost skalarnog množenja.
  Dakle, $\mathcal{W}$ je vektorski potprostor.

  Obrnuto, ako je $\mathcal{W}$ vektorski potprostor onda očigledno važi iz definicija vektorskog sabiranja i skalarnog množenja da je $\alpha x + \beta y \in \mathcal{W}$.
\end{proof}

\begin{definition}[Linearna kombinancija]
  Neka je $\mathcal{V}$ vektorski prostor nad poljem $\mathbb{F}$ i neka su $\upsilon_1, \upsilon_2, \ldots, \upsilon_n \in \mathcal{V}$, $\alpha_1, \alpha_2, \ldots, \alpha_n \in \mathbb{F}$.
  Tada se suma $\alpha_1 \upsilon_1 + \alpha_2 \upsilon_2 + \ldots + \alpha_n \upsilon_n$ naziva \textit{linearnom kombinacijom} vektora $\upsilon_1, \upsilon_2, \ldots, \upsilon_n \in \mathcal{V}$.
\end{definition}

\begin{definition}[Linearna zavisnost i nezavisnost]
  Konačan skup vektora $\{\upsilon_i\}$ je \textit{linearno zavisan} ako postoji odgovarajući skup skalara $\{\alpha_i\}$, ne svih nula, tako da
  \[\sum_{i} \alpha_i \upsilon_i = 0.\]
  S druge strane, ako iz $\sum_{i} \alpha_i \upsilon_i = 0$ sledi da je $\alpha_i = 0$ za svako $i$, onda se kaže da je skup $\{\upsilon_i\}$ \textit{linearno nezavisan}.
  
  Za beskonačne skupove vektora kažemo da su linearno nezavisni ako su im svi konačni podskupovi linearno nezavisni.
  U suprotnom kažemo da su linearno zavisni.
\end{definition}

\begin{corollary}
  \begin{enumerate}
    \item Svaki skup koji sadrži linearno zavisan skup je linearno zavisan (za sve ostale element stavimo $\alpha_i = 0$).
    \item Svaki podskup linearno nezavisnog skupa je linearno nezavisan (kontrapositiv prve posledice).
    \item Bilo koji skup koji sadrži $0$ vektor je linearno zavisan.
  \end{enumerate}
\end{corollary}

\begin{example}
  \begin{enumerate}
    \item U realnom vektorskom prostor $\R^2$ kao primere uzmimo sledeće vektore
      \begin{itemize}
        \item Nezavisni: $\{(1, 0), (0, 1)\}$, $\{(2, 2)\}$
        \item Zavisni: $\{(1, 2), (2, 4)\}$, $\{(1, 0), (0, 1), (1, 1)\}$
      \end{itemize}
    \item Slično, za realni vektorski prostor $\R^3$ uzmimo vektore
      \begin{itemize}
        \item Nezavisni: $\{(1, 0, 0), (0, 1, 0), (0, 0, 1)\}$, $\{(1, 0, 1), (1, 1, 0), (0, 1, 1)\}$
        \item Zavisni: $\{(0, 0, 0), (1, 0, 0), (0, 1, 0)\}$, $\{(1, 0, 0), (1, 1, 0), (0, 1, 0)\}$
      \end{itemize}
    \item Postmarajući skup $\C^2$ kao vektorski prostor nad poljem $\C$ imamo
      \begin{itemize}
        \item Nezavisni: $\{(1, 0), (0, 1)\}$, $\{(i, 1), (1, i)\}$
        \item Zavisni: $\{(1, 1), (i, i)\}$, $\{(1, 2), (2, 4)\}$
      \end{itemize}
    \item Konačno, razmatrajući skup $\C^3$ kao kompleksan vektorski prostor imamo
      \begin{itemize}
        \item Nezavisni: $\{(1, 0, 0), (0, 1, 0), (0, 0, 1)\}$, $\{(5, 0, 0), (1, i, 0), (1, 2, 3)\}$
        \item Zavisni: $\{(1, 1, 1), (2, 2, 2), (0, 1, 1)\}$, $\{(0, 1, i), (i, 0, 1), (i, 1, 1 + i)\}$
      \end{itemize}
  \end{enumerate}
\end{example}

\begin{theorem}
  Skup nenula vektora $\upsilon_1, \ldots, \upsilon_n$ je linearno zavisan ako i samo ako je neki vektor $\upsilon_i$, $2 \le i \le n$, linearna kombinacija prethodnih.
\end{theorem}

\begin{proof}
  Prvo pretpostavimo da su vektori $\upsilon_1 , \ldots, \upsilon_n$ linearno zavisni.
  Neka je $k$ najmanji prirodni broj takav da je $\upsilon_1 , \ldots, \upsilon_k$ linearno zavisan skup.
  Onda imam 
  \[\alpha_1 \upsilon_1 + \ldots + \alpha_k \upsilon_k = 0\]
  za odgovarajući skup ne svih nula $\alpha_i$.
  Po definiciji $k$ očigledno je $\alpha_k \neq 0$.
  Dakle, imamo
  \[\upsilon_k = \frac{-\alpha_1}{\alpha_k} \upsilon_1 + \ldots + \frac{\alpha_{k - 1}}{\alpha_k} \upsilon_k.\]

  Obrnuto važi direktno iz činjenice da svaki skup koji sadrži linearno zavisan skup je i sam linearno zavisan.
\end{proof}

\begin{definition}[Baza]
  Skup linearno nezavisnih vektora koji generišu ceo vektorski prostor nazivamo \textit{bazom}.
\end{definition}

\begin{theorem}
  Svaki vektorski prostor ima bazu.
\end{theorem}

\begin{proof}
  Daćemo samo skicu dokaza ovde.

  Dokaz ove teoreme se zasniva na Zornovoj lemi koja je ekvivalentna aksiomi izbora i glasi da ako svaki lanac (totalno uredjeni podskup) u parcijalno uredjenom skupu ima majorantu onda to uredjenje ima maksimalni element.
  Za dati vektorski prostor postmatramo skup svih linearno nezavisnih podskupova i parcijalo uredjenje definisano inkluzijom.
  Odatle dokazujem da svaki totalno uredjeni podskup našeg skupa ima majorantu u formi unije svih elemenata podskupa.
  Dakle, prema Zornovoj lemi znamo da naš skup ima maksimalni element, koji lako dokazujemo da je zapravo baza našeg vektorskog prostora.
\end{proof}

\begin{theorem}
  Broj elemenata bilo koje baze u konačno dimenzionom vektorskom prostoru je isti kao i kod bilo koje druge baze.
\end{theorem}

\begin{proof}
  Prvo ćemo dokazati da ako imamo bazu sa $n$ elemenata, onda je svaki skup od $m \ge n$ vektora linearno zavisan.
  
  Neka je $\{\upsilon_1, \ldots, \upsilon_n\}$ baza vektorskog prostora $\mathcal{V}$ nad $mathbb{F}$ i neka je $\mathcal{S} \subset \mathcal{V}$ podskup sa $m \ge n$ vektora.
  Kako baza generiše vektorski prostor znamo da se svaki vektor $x_j \in \mathcal{S}$ može napisati kao
  \[\sum_{i = 1}^{n} \alpha_{ij} \upsilon_i, \text{for some} \alpha_i \in \mathbb{F}.\]
  Sada imamo
  \[\sum_{j = 1}^{m} \beta_j x_j = \sum_{j = 1}^{m} \beta_j \sum_{i = 1}^{n} \alpha_{ij} \upsilon_i = \sum_{i = 1}^{n} (\sum_{j = 1}^{n} (\alpha_{ij} \beta_j)) \upsilon_i.\]
  Na kraju imamo homogeni linearni sistem sa $p > n$ nepoznatih, koji ima beskonačno mnogo rešenja.
  Dakle, $\mathcal{S}$ je linearno zavisan.

  Odatle ako imamo dve baze sa $n$ i $m$ elemenata očigledno sledi da je $m \ge n$ i $n \ge m$, pa imamo $n = m$.
\end{proof}

\begin{theorem}
  Neka je $\mathcal{V}$ vektorski prostor nad poljem $\mathbb{F}$ i neka je $\{\upsilon_1, \upsilon_2, \ldots, \upsilon_n\}$ baza $\mathcal{V}$.
  Svaki vektor $x \in \mathcal{V}$ može da se na jedinstven način predstavi kao linearna kombinacija baznih vektora.
\end{theorem}

\begin{proof}
  Očigledno da svaki vektor može da se predstavi kao linearna kombinacija jer baza generiše vektorski prostor.

  Ostaje nam samo jedinstvenost da dokažemo.
  Neka su $x = \alpha_1 \upsilon_1 + \ldots \alpha_n \upsilon_n$ i $x = \beta_1 \upsilon_1 + \ldots + \beta_n \upsilon_n$ dve representacije vektora $x$ preko baznih vektora.
  Kada bi oduzeli ove jednu od druge imali bi $0 = (\alpha_1 - \beta_1) \upsilon_1 + \ldots + (\alpha_n - \beta_n) \upsilon_n$.
  Kako je baza linearno nezavisna, svi koeficijenti $\alpha_1 - \beta_1, \ldots, \alpha_n - \beta_n$ moraju biti jednaki nuli.
  Dakle ove representacije su jednake.
\end{proof}

\begin{example}
  \begin{enumerate}
    \item Posmatrajmo prvo realan vektorski prostor $\R^2$ sa kanonskom bazom i vektor $(4, 2)$.
      Trivijalno ovaj vektor možemo da razložimo na $(4, 2) = 4 \cdot (1, 0) + 2 \cdot (0, 1)$.
    \item Posmatrajmo isti vektor i vektorski prostor samo ovoga puta sa bazom $\{(2, -1), (-4, 3)\}$.
      Sada imamo $(4, 2) = \alpha_1 (2, -1) + \alpha_2 (-4, 3)$.
      Odavde lako dobijamo $(4, 2) = (2 \alpha_1 - 4 \alpha_2, -\alpha_1 + 3 \alpha_2)$.
      Dakle imamo sistem $4 = 2 \alpha_1 - 4 \alpha_2$ i $2 = -\alpha_1 + 3 \alpha_2$.
      Rešavanjen sistema, na primer eliminacijom, dobijamo $(4, 2) = 10 \cdot (2, -1) + 4 \cdot (-4, 3)$.
    \item U generalnom slučaju može da pratimo istu proceduru rešavanja sistema kao u primeru (2), zato što je svaki n-dimenzioni vektorski prostor nad $\mathbb{F}$ isomorfan sa $\mathbb{F}^n$.
  \end{enumerate}
\end{example}

\end{document}
