\documentclass{article}

\input{../preamble.tex}

\title{Kvadratne matrice kao linearna preslikavanja na vektorskom prostoru}
\author{Mihailo Đurić}
\date{\today}

\begin{document}

\maketitle
\tableofcontents
\newpage

\section{Sistemi linearno zavisnih i nezavisnih vektora}

\begin{definition}[Vektorski prostor]
  \textit{Vektorski prostor nad poljem $\mathbb{F}$ (elementi $\mathbb{F}$ zovu se skalari)} je skup $\mathcal{V}$ (elementi $\mathcal{V}$ zovu se vektori) sa binarnom operacijom vektorskog sabiranja i binarnom funkcijom skalarnog množenja.

  Vektorsko sabiranje je binarna operacija $+ : \mathcal{V} \times \mathcal{V} \to \mathcal{V}$ koja svakom paru vektora $x, y \in \mathcal{V}$ dodeljuje vektor $x + y \in \mathcal{V}$ tako da
  \begin{enumerate}
    \item Sabiranje je komutativno, $x + y = y + x$;
    \item Sabiranje je asocijativno, $x + (y + z) = (x + y) + z$;
    \item Postoji jedinstven vektor $0 \in \mathcal{V}$ tako da $x + 0 = x$ za svaki vektor $x$;
    \item Za svaki vektor $x \in \mathcal{V}$ postoji odgovarajući vektor $-x$ tako da $x + (-x) = 0$.
  \end{enumerate}
  
  Skalarno množenje je binarna funkcija $\cdot : \mathbb{F} \times \mathcal{V} \to \mathcal{V}$ koja svakom paru $\alpha \in \mathbb{F}$ i $x \in \mathcal{V}$ dodeljuje vektor $\alpha x \in \mathcal{V}$ tako da
  \begin{enumerate}
    \item Skalarno množenje je asocijativno, $\alpha (\beta x) = (\alpha \beta) x$;
    \item $1 \cdot x = x$ za svaki vektor $x$;
    \item Skalarno množenje je distributivno prema vektorskom sabiranju, $\alpha (x + y) = \alpha x + \alpha y$;
    \item Skalarno množenje je distributivno prema sabiranju skalara, $(\alpha + \beta) x = \alpha x + \beta x$.
  \end{enumerate}
\end{definition}

\begin{definition}[Vektorski potprostor]
  \textit{Potprostor} $\mathcal{W}$ je podskup vektorskog prostora $\mathcal{V}$ koji je i sam vektorski prostor nad poljem $\mathbb{F}$ sa istim operacijama kao i $\mathcal{V}$
\end{definition}

\begin{theorem}
  Neprazan skup $\mathcal{W} \subset \mathcal{V}$ je potprostor vektorskog prostora $\mathcal{V}$ nad poljem $\mathbb{F}$ ako i samo ako za svaki $w_1, w_2 \in \mathcal{W}$ i $\alpha, \beta \in \mathbb{F}$, sledi da je $\alpha w_1 + \beta w_2 \in \mathcal{W}$.
\end{theorem}

\begin{proof}
  Neka je $\mathcal{W}$ neprazan podskup $\mathcal{V}$ tako da važi da za svaki $w_1, w_2 \in \mathcal{W}$ i $\alpha, \beta \in \mathbb{F}$, sledi da je $\alpha w_1 + \beta w_2 \in \mathcal{W}$.
  Za bilo koji vektor $x \in \mathcal{W}$ važi $(-1) x + x = 0$, odakle $0 \in \mathcal{W}$.
  Ako je $x \in \mathcal{V}$ i $\alpha \in \mathbb{F}$ onda znamo da je $\alpha x + 0 = \alpha x$, dakle $\alpha x \in \mathcal{W}$.
  Specifično, $(-1) x = -x$ je u $\mathcal{W}$.
  Konačno, ako su $x, y \in \mathcal{W}$ onda je $(1) x + (1) y = x + y$ isto u $\mathcal{W}$.
  Asocijativnosti i komutativnost sabiranja se nasledjuje iz $\mathcal{V}$, kao i asocijativnost i distributivnost skalarnog množenja.
  Dakle, $\mathcal{W}$ je vektorski potprostor.

  Obrnuto, ako je $\mathcal{W}$ vektorski potprostor onda očigledno važi iz definicija vektorskog sabiranja i skalarnog množenja da je $\alpha x + \beta y \in \mathcal{W}$.
\end{proof}

\begin{definition}[Linearna kombinancija]
  Neka je $\mathcal{V}$ vektorski prostor nad poljem $\mathbb{F}$ i neka su $\upsilon_1, \upsilon_2, \ldots, \upsilon_n \in \mathcal{V}$, $\alpha_1, \alpha_2, \ldots, \alpha_n \in \mathbb{F}$.
  Tada se suma $\alpha_1 \upsilon_1 + \alpha_2 \upsilon_2 + \ldots + \alpha_n \upsilon_n$ naziva \textit{linearnom kombinacijom} vektora $\upsilon_1, \upsilon_2, \ldots, \upsilon_n \in \mathcal{V}$.
\end{definition}

\begin{definition}[Linearna zavisnost i nezavisnost]
  Konačan skup vektora $\{\upsilon_i\}$ je \textit{linearn zavisan} ako postoji odgovarajući skup skalara $\{\alpha_i\}$, ne svih nula, tako da
  \[\sum_{i} \alpha_i \upsilon_i = 0.\]
  S druge strane, ako iz $\sum_{i} \alpha_i \upsilon_i = 0$ sledi da je $\alpha_i = 0$ za svako $i$, onda se kaže da je skup $\{\upsilon_i\}$ \textit{linearno nezavisan}.
  
  Za beskonačne skupove vektora kažemo da su linearno nezavisni ako su im svi konačni podskupovi linearno nezavisni.
  U suprotnom kažemo da su linearno zavisni.
\end{definition}

\begin{corollary}
  \begin{enumerate}
    \item Svaki skup koji sadrži linearno zavisan skup je linearno zavisan (za sve ostale element stavimo $\alpha_i = 0$).
    \item Svaki podskup linearno nezavisnog skupa je linearno nezavisan (kontrapositiv prve posledice).
    \item Bilo koji skup koji sadrži $0$ vektor je linearno zavisan.
  \end{enumerate}
\end{corollary}

\begin{example}
  \begin{enumerate}
    \item U realnom vektorskom prostor $\R^2$ kao primere uzmimo sledeće vektore
      \begin{itemize}
        \item Nezavisni: $\{(1, 0), (0, 1)\}$, $\{(2, 2)\}$
        \item Zavisni: $\{(1, 2), (2, 4)\}$, $\{(1, 0), (0, 1), (1, 1)\}$
      \end{itemize}
    \item Slično, za realni vektorski prostor $\R^3$ uzmimo vektore
      \begin{itemize}
        \item Nezavisni: $\{(1, 0, 0), (0, 1, 0), (0, 0, 1)\}$, $\{(1, 0, 1), (1, 1, 0), (0, 1, 1)\}$
        \item Zavisni: $\{(0, 0, 0), (1, 0, 0), (0, 1, 0)\}$, $\{(1, 0, 0), (1, 1, 0), (0, 1, 0)\}$
      \end{itemize}
    \item Postmarajući skup $\C^2$ kao vektorski prostor nad poljem $\C$ imamo
      \begin{itemize}
        \item Nezavisni: $\{(1, 0), (0, 1)\}$, $\{(i, 1), (1, i)\}$
        \item Zavisni: $\{(1, 1), (i, i)\}$, $\{(1, 2), (2, 4)\}$
      \end{itemize}
    \item Konačno, razmatrajući skup $\C^3$ kao kompleksan vektorski prostor imamo
      \begin{itemize}
        \item Nezavisni: $\{(1, 0, 0), (0, 1, 0), (0, 0, 1)\}$, $\{(5, 0, 0), (1, i, 0), (1, 2, 3)\}$
        \item Zavisni: $\{(1, 1, 1), (2, 2, 2), (0, 1, 1)\}$, $\{(0, 1, i), (i, 0, 1), (i, 1, 1 + i)\}$
      \end{itemize}
  \end{enumerate}
\end{example}

\begin{theorem}
  Skup nenula vektora $\upsilon_1, \ldots, \upsilon_n$ je linearno zavisan ako i samo ako je neki vektor $\upsilon_i$, $2 \le i \le n$, linearna kombinacija prethodnih.
\end{theorem}

\begin{proof}
  Prvo pretpostavimo da su vektori $\upsilon_1 , \ldots, \upsilon_n$ linearno zavisni.
  Neka je $k$ najmanji prirodni broj takav da je $\upsilon_1 , \ldots, \upsilon_k$ linearno zavisan skup.
  Onda imam 
  \[\alpha_1 \upsilon_1 + \ldots + \alpha_k \upsilon_k = 0\]
  za odgovarajući skup ne svih nula $\alpha_i$.
  Po definiciji $k$ očigledno je $\alpha_k \neq 0$.
  Dakle, imamo
  \[\upsilon_k = \frac{-\alpha_1}{\alpha_k} \upsilon_1 + \ldots + \frac{\alpha_{k - 1}}{\alpha_k} \upsilon_k.\]

  Obrnuto važi direktno iz činjenice da svaki skup koji sadrži linearno zavisan skup je i sam linearno zavisan.
\end{proof}

\begin{definition}[Baza]
  Skup linearno nezavisnih vektora koji generišu ceo vektorski prostor nazivamo \textit{bazom}.
\end{definition}

\begin{theorem}
  Svaki vektorski prostor ima bazu.
\end{theorem}

\begin{proof}
  Daćemo samo skicu dokaza ovde.
  Dokaz ove teoreme se zasniva na Zornovoj lemi koja je ekvivalentna aksiomi izbora i glasi da ako svaki lanac (totalno uredjeni podskup) u parcijalno uredjenom skupu ima majorantu onda to uredjenje ima maksimalni element.
  Za dati vektorski prostor postmatramo skup svih linearno nezavisnih podskupova i parcijalo uredjenje definisano inkluzijom.
  Odatle dokazujem da svaki totalno uredjeni podskup našeg skupa ima majorantu u formi unije svih elemenata podskupa.
  Dakle, prema Zornovoj lemi znamo da naš skup ima maksimalni element, koji lako dokazujemo da je zapravo baza našeg vektorskog prostora.
\end{proof}


\end{document}
