\documentclass{article}

% ===================================================================
% PREAMBLE FOR MATHEMATICS EXERCISE NOTEBOOK
% ===================================================================
% This file contains all the packages and custom commands for the document.
% Keeping it separate from main.tex makes the project cleaner.


% -------------------------------------------------------------------
% DOCUMENT & ENCODING
% -------------------------------------------------------------------
\usepackage[utf8]{inputenc} % Allows you to type characters like á, ö, etc. directly
\usepackage[T1]{fontenc}    % Specifies font encoding, improves font rendering and hyphenation


% -------------------------------------------------------------------
% PAGE LAYOUT & GEOMETRY
% -------------------------------------------------------------------
\usepackage{geometry}
\geometry{
  a4paper,         % Or letterpaper, etc.
  total={170mm,257mm},
  left=20mm,
  top=20mm,
}


% -------------------------------------------------------------------
% CORE MATH PACKAGES (AMS - American Mathematical Society)
% -------------------------------------------------------------------
\usepackage{amsmath}    % The premier package for typesetting math equations
\usepackage{amssymb}    % Provides lots of extra math symbols (like \mathbb)
\usepackage{amsthm}     % Comprehensive theorem-like environments
\usepackage{mathtools}  % An extension of amsmath, provides more tools and fixes


% -------------------------------------------------------------------
% UTILITY & FORMATTING PACKAGES
% -------------------------------------------------------------------
\usepackage{graphicx}      % For including images (\includegraphics)
\usepackage{xcolor}        % For defining and using colors
\usepackage{booktabs}      % For creating beautiful, professional-looking tables (\toprule, \midrule, \bottomrule)
\usepackage{siunitx}       % For typesetting numbers and units consistently
\usepackage{enumitem}      % Provides more control over list environments (itemize, enumerate)


% -------------------------------------------------------------------
% HYPERLINKS & CROSS-REFERENCING
% -------------------------------------------------------------------
\usepackage{hyperref}
\hypersetup{
    colorlinks=true,       % false: boxed links; true: colored links
    linkcolor=teal,        % color of internal links (e.g., sections)
    citecolor=green,       % color of links to bibliography
    filecolor=magenta,     % color of file links
    urlcolor=blue          % color of external links
}


% -------------------------------------------------------------------
% THEOREM & DEFINITION ENVIRONMENTS
% -------------------------------------------------------------------
% This section sets up consistent numbering and styling for problems,
% definitions, theorems, etc.

\theoremstyle{definition} % Use a style that is less flashy than the default "plain" style
\newtheorem{definition}{Definition}[section] % Number problems as Problem X.Y (where X is the section number)
\newtheorem{example}[definition]{Example}

\theoremstyle{plain}
\newtheorem{theorem}[definition]{Theorem}     % Share the same counter as 'problem'
\newtheorem{lemma}[definition]{Lemma}
\newtheorem{corollary}[definition]{Corollary}

\theoremstyle{remark}
\newtheorem{remark}[definition]{Remark}

\renewcommand{\proofname}{Proof}
% -------------------------------------------------------------------
% CUSTOM MATH COMMANDS (MACROS)
% -------------------------------------------------------------------
% Define shortcuts for commonly used mathematical notation to save
% time and ensure consistency.

% Sets of numbers
\newcommand{\R}{\mathbb{R}} % Real numbers
\newcommand{\C}{\mathbb{C}} % Complex numbers
\newcommand{\N}{\mathbb{N}} % Natural numbers
\newcommand{\Z}{\mathbb{Z}} % Integers
\newcommand{\Q}{\mathbb{Q}} % Rational numbers

% Calculus operators
\newcommand{\dd}{\, \mathrm{d}} % For integrals, e.g., \int f(x)\dd x
\newcommand{\pdv}[2]{\frac{\partial #1}{\partial #2}} % Partial derivative
\newcommand{\dv}[2]{\frac{\mathrm{d} #1}{\mathrm{d} #2}} % Full derivative

% Linear Algebra
\DeclareMathOperator{\Tr}{Tr} % Trace of a matrix
\newcommand{\T}{\mathsf{T}}   % Transpose, e.g., A^\T

% Probability & Statistics
\DeclareMathOperator{\E}{\mathbb{E}} % Expectation
\DeclareMathOperator{\Var}{Var} % Variance
\DeclareMathOperator{\Cov}{Cov} % Covariance


% ===================================================================
% END OF PREAMBLE
% ===================================================================


\title{Kvadratne matrice kao linearna preslikavanja na vektorskom prostoru}
\author{Mihailo Đurić}
\date{\today}

\begin{document}

\maketitle
\tableofcontents
\newpage

\section{Sistemi linearno zavisnih i nezavisnih vektora}

\begin{definition}[Vektorski prostor]
  \textit{Vektorski prostor nad poljem $\mathbb{F}$} (elementi $\mathbb{F}$ zovu se skalari) je skup $\mathcal{V}$ (elementi $\mathcal{V}$ zovu se vektori) sa binarnom operacijom vektorskog sabiranja i binarnom funkcijom skalarnog množenja.

  Vektorsko sabiranje je binarna operacija $+ : \mathcal{V} \times \mathcal{V} \to \mathcal{V}$ koja svakom paru vektora $x, y \in \mathcal{V}$ dodeljuje vektor $x + y \in \mathcal{V}$ tako da
  \begin{enumerate}
    \item Sabiranje je komutativno, $x + y = y + x$;
    \item Sabiranje je asocijativno, $x + (y + z) = (x + y) + z$;
    \item Postoji jedinstven vektor $0 \in \mathcal{V}$ tako da $x + 0 = x$ za svaki vektor $x$;
    \item Za svaki vektor $x \in \mathcal{V}$ postoji odgovarajući vektor $-x \in \mathcal{V}$ tako da $x + (-x) = 0$.
  \end{enumerate}
  
  Skalarno množenje je binarna funkcija $\cdot : \mathbb{F} \times \mathcal{V} \to \mathcal{V}$ koja svakom paru $\alpha \in \mathbb{F}$ i $x \in \mathcal{V}$ dodeljuje vektor $\alpha x \in \mathcal{V}$ tako da
  \begin{enumerate}
    \item Skalarno množenje je asocijativno, $\alpha (\beta x) = (\alpha \beta) x$;
    \item $1 \cdot x = x$ za svaki vektor $x \in \mathcal{V}$;
    \item Skalarno množenje je distributivno prema vektorskom sabiranju, $\alpha (x + y) = \alpha x + \alpha y$;
    \item Skalarno množenje je distributivno prema sabiranju skalara, $(\alpha + \beta) x = \alpha x + \beta x$.
  \end{enumerate}
\end{definition}

Intuitivno možemo da posmatramo vektorski prostor kao algebrasku strukturu linearnosti (linearnih kombinacija).

\begin{definition}[Vektorski potprostor]
  \textit{Potprostor} $\mathcal{W}$ je podskup vektorskog prostora $\mathcal{V}$ koji je i sam vektorski prostor nad poljem $\mathbb{F}$ sa istim operacijama kao i $\mathcal{V}$
\end{definition}

\begin{theorem}
  Neprazan skup $\mathcal{W} \subset \mathcal{V}$ je potprostor vektorskog prostora $\mathcal{V}$ nad poljem $\mathbb{F}$ ako i samo ako za svaki $x, y \in \mathcal{W}$ i $\alpha, \beta \in \mathbb{F}$, sledi da je $\alpha x + \beta y \in \mathcal{W}$.
\end{theorem}

\begin{proof}
  Neka je $\mathcal{W}$ neprazan podskup $\mathcal{V}$ tako da važi da za svaki $x, y \in \mathcal{W}$ i $\alpha, \beta \in \mathbb{F}$, sledi da je $\alpha x + \beta y \in \mathcal{W}$.
  Za bilo koji vektor $x \in \mathcal{W}$ važi $(-1) x + x = 0$, odakle $0 \in \mathcal{W}$.
  Ako je $x \in \mathcal{W}$ i $\alpha \in \mathbb{F}$ onda znamo da je $\alpha x + 0 = \alpha x$, dakle $\alpha x \in \mathcal{W}$.
  Specifično, $(-1) x = -x$ je u $\mathcal{W}$.
  Konačno, ako su $x, y \in \mathcal{W}$ onda je $(1) x + (1) y = x + y$ isto u $\mathcal{W}$.
  Asocijativnosti i komutativnost sabiranja se nasledjuje iz $\mathcal{V}$, kao i asocijativnost i distributivnost skalarnog množenja.
  Dakle, $\mathcal{W}$ je vektorski potprostor.

  Obrnuto, ako je $\mathcal{W}$ vektorski potprostor onda očigledno važi iz definicija vektorskog sabiranja i skalarnog množenja da je $\alpha x + \beta y \in \mathcal{W}$.
\end{proof}

\begin{definition}[Linearna kombinancija]
  Neka je $\mathcal{V}$ vektorski prostor nad poljem $\mathbb{F}$ i neka su $x_1, x_2, \ldots, x_n \in \mathcal{V}$, $\alpha_1, \alpha_2, \ldots, \alpha_n \in \mathbb{F}$.
  Tada se suma $\alpha_1 x_1 + \alpha_2 x_2 + \ldots + \alpha_n x_n$ naziva \textit{linearnom kombinacijom} vektora $x_1, x_2, \ldots, x_n \in \mathcal{V}$.
\end{definition}

\begin{definition}[Linearna zavisnost i nezavisnost]
  Konačan skup vektora $\{x_i\}$ je \textit{linearno zavisan} ako postoji odgovarajući skup skalara $\{\alpha_i\}$, ne svih nula, tako da
  \[\sum_{i} \alpha_i x_i = 0.\]
  S druge strane, ako iz $\sum_{i} \alpha_i x_i = 0$ sledi da je $\alpha_i = 0$ za svako $i$, onda se kaže da je skup $\{x_i\}$ \textit{linearno nezavisan}.
  
  Za beskonačne skupove vektora kažemo da su linearno nezavisni ako su im svi konačni podskupovi linearno nezavisni.
  U suprotnom kažemo da su linearno zavisni.
\end{definition}

\begin{corollary}
  \begin{enumerate}
    \item Svaki skup koji sadrži linearno zavisan skup je linearno zavisan (za sve ostale element stavimo $\alpha_i = 0$).
    \item Svaki podskup linearno nezavisnog skupa je linearno nezavisan (kontrapositiv prve posledice).
    \item Bilo koji skup koji sadrži $0$ vektor je linearno zavisan.
    \item Svaki skup sa samo jednim nenula vektorom je linearno nezavisan.
  \end{enumerate}
\end{corollary}

\begin{example}
  \begin{enumerate}
    \item U realnom vektorskom prostor $\R^2$ kao primere uzmimo sledeće vektore
      \begin{itemize}
        \item Nezavisni: $\{(1, 0), (0, 1)\}$, $\{(2, 2)\}$
        \item Zavisni: $\{(1, 2), (2, 4)\}$, $\{(1, 0), (0, 1), (1, 1)\}$
      \end{itemize}
    \item Slično, za realni vektorski prostor $\R^3$ uzmimo vektore
      \begin{itemize}
        \item Nezavisni: $\{(1, 0, 0), (0, 1, 0), (0, 0, 1)\}$, $\{(1, 0, 1), (1, 1, 0), (0, 1, 1)\}$
        \item Zavisni: $\{(0, 0, 0), (1, 0, 0), (0, 1, 0)\}$, $\{(1, 0, 0), (1, 1, 0), (0, 1, 0)\}$
      \end{itemize}
    \item Postmarajući skup $\C^2$ kao vektorski prostor nad poljem $\C$ imamo
      \begin{itemize}
        \item Nezavisni: $\{(1, 0), (0, 1)\}$, $\{(i, 1), (1, i)\}$
        \item Zavisni: $\{(1, 1), (i, i)\}$, $\{(1, 2), (2, 4)\}$
      \end{itemize}
    \item Konačno, razmatrajući skup $\C^3$ kao kompleksan vektorski prostor imamo
      \begin{itemize}
        \item Nezavisni: $\{(1, 0, 0), (0, 1, 0), (0, 0, 1)\}$, $\{(5, 0, 0), (1, i, 0), (1, 2, 3)\}$
        \item Zavisni: $\{(1, 1, 1), (2, 2, 2), (0, 1, 1)\}$, $\{(0, 1, i), (i, 0, 1), (i, 1, 1 + i)\}$
      \end{itemize}
  \end{enumerate}
\end{example}

\begin{theorem}
  Skup nenula vektora $x_1, \ldots, x_n$ je linearno zavisan ako i samo ako je neki vektor $x_i$, $2 \le i \le n$, linearna kombinacija prethodnih.
\end{theorem}

\begin{proof}
  Prvo pretpostavimo da su vektori $x_1 , \ldots, x_n$ linearno zavisni.
  Neka je $k$ najmanji prirodni broj takav da je $x_1 , \ldots, x_k$ linearno zavisan skup.
  Onda imam 
  \[\alpha_1 x_1 + \ldots + \alpha_k x_k = 0\]
  za odgovarajući skup ne svih nula $\alpha_i$.
  Po definiciji $k$ očigledno je $\alpha_k \neq 0$.
  Dakle, imamo
  \[x_k = \frac{-\alpha_1}{\alpha_k} x_1 + \ldots + \frac{-\alpha_{k - 1}}{\alpha_k} x_k.\]

  Obrnuto važi direktno iz činjenice da svaki skup koji sadrži linearno zavisan skup je i sam linearno zavisan.
\end{proof}

\begin{definition}[Baza]
  Skup linearno nezavisnih vektora koji generišu ceo vektorski prostor nazivamo \textit{bazom}.
\end{definition}

\begin{theorem}
  Svaki vektorski prostor ima bazu.
\end{theorem}

\begin{proof}
  Daćemo samo skicu dokaza ovde.

  Dokaz ove teoreme se zasniva na Zornovoj lemi koja je ekvivalentna aksiomi izbora i glasi da ako svaki lanac (totalno uredjeni podskup) u parcijalno uredjenom skupu ima majorantu onda to uredjenje ima maksimalni element.
  Za dati vektorski prostor postmatramo skup svih linearno nezavisnih podskupova i parcijalo uredjenje definisano inkluzijom.
  Odatle dokazujem da svaki totalno uredjeni podskup našeg skupa ima majorantu u formi unije svih elemenata podskupa.
  Dakle, prema Zornovoj lemi znamo da naš skup ima maksimalni element, koji lako dokazujemo da je zapravo baza našeg vektorskog prostora.
\end{proof}

\begin{theorem}
  Broj elemenata bilo koje baze u konačno dimenzionom vektorskom prostoru je isti kao i kod bilo koje druge baze.
\end{theorem}

\begin{proof}
  Prvo ćemo dokazati da ako imamo bazu sa $n$ elemenata, onda je svaki skup od $m > n$ vektora linearno zavisan.
  
  Neka je $\{x_1, \ldots, x_n\}$ baza vektorskog prostora $\mathcal{V}$ nad $\mathbb{F}$ i neka je $\mathcal{S} \subset \mathcal{V}$ podskup sa $m > n$ vektora.
  Kako baza generiše vektorski prostor znamo da se svaki vektor $v_j \in \mathcal{S}$ može napisati kao
  \[\sum_{i = 1}^{n} \alpha_{ij} x_i, \text{ za neke } \alpha_i \in \mathbb{F}.\]
  Sada imamo
  \[\sum_{j = 1}^{m} \beta_j v_j = \sum_{j = 1}^{m} \beta_j \sum_{i = 1}^{n} \alpha_{ij} x_i = \sum_{i = 1}^{n} (\sum_{j = 1}^{m} (\alpha_{ij} \beta_j)) x_i.\]
  Na kraju imamo homogeni linearni sistem sa $p > n$ nepoznatih, koji ima beskonačno mnogo rešenja.
  Dakle, $\mathcal{S}$ je linearno zavisan.

  Odatle ako imamo dve baze sa $n$ i $m$ elemenata očigledno sledi da je $m \ge n$ i $n \ge m$, pa imamo $n = m$.
\end{proof}

\begin{theorem}
  Neka je $\mathcal{V}$ vektorski prostor nad poljem $\mathbb{F}$ i neka je $\{x_1, x_2, \ldots, x_n\}$ baza $\mathcal{V}$.
  Svaki vektor $x \in \mathcal{V}$ može da se na jedinstven način predstavi kao linearna kombinacija baznih vektora.
\end{theorem}

\begin{proof}
  Očigledno da svaki vektor može da se predstavi kao linearna kombinacija jer baza generiše vektorski prostor.

  Ostaje nam samo jedinstvenost da dokažemo.
  Neka je $v = \alpha_1 x_1 + \ldots \alpha_n x_n$ i $ v = \beta_1 x_1 + \ldots + \beta_n x_n$ dve representacije vektora $v$ preko baznih vektora.
  Kada bi oduzeli jednu od druge imali bi $0 = (\alpha_1 - \beta_1) x_1 + \ldots + (\alpha_n - \beta_n) x_n$.
  Kako je baza linearno nezavisna, svi koeficijenti $\alpha_1 - \beta_1, \ldots, \alpha_n - \beta_n$ moraju biti jednaki nuli.
  Dakle ove representacije su jednake.
\end{proof}

Jedinstvene skalare koji nam daju representaciju vektora preko odredjene baze nazivamo koordinatama vektora.

\begin{example}
  \begin{enumerate}
    \item Posmatrajmo prvo realan vektorski prostor $\R^2$ sa kanonskom bazom i vektor $(4, 2)$.
      Trivijalno ovaj vektor možemo da razložimo na $(4, 2) = 4 \cdot (1, 0) + 2 \cdot (0, 1)$.
    \item Posmatrajmo isti vektor i vektorski prostor samo ovoga puta sa bazom $\{(2, -1), (-4, 3)\}$.
      Sada imamo $(4, 2) = \alpha_1 (2, -1) + \alpha_2 (-4, 3)$.
      Odavde lako dobijamo $(4, 2) = (2 \alpha_1 - 4 \alpha_2, -\alpha_1 + 3 \alpha_2)$.
      Dakle imamo sistem $4 = 2 \alpha_1 - 4 \alpha_2$ i $2 = -\alpha_1 + 3 \alpha_2$.
      Rešavanjen sistema, na primer eliminacijom, dobijamo $(4, 2) = 10 \cdot (2, -1) + 4 \cdot (-4, 3)$.
    \item U generalnom slučaju može da pratimo istu proceduru rešavanja sistema kao u primeru (2), zato što je svaki n-dimenzioni vektorski prostor nad $\mathbb{F}$ isomorfan sa $\mathbb{F}^n$.
  \end{enumerate}
\end{example}

\section{Kvadratne matrice kao linearna preslikavanja}

\begin{theorem}
  Svaka kvadratna matrica može da se sagleda kao linearno preslikavanje nad datim vektorskim prostorom tako što bazne vektore pomnožimo datom matricom i dobijemo nove bazne vektore.
\end{theorem}

Ovo je posledica generalnije ideje koju ćemo dokazati kasnije.

\begin{theorem}
  Linearno preslikavanja $L$ salje linearno nezavisne vektore u linearno nezavisne ako i samo ako je $L$ injektivna.
\end{theorem}

\begin{proof}
  Neka je $L$ injektivna linearna transformacija i neka je $S$ linearno nezavisan podskup sa $n$ elemenata vektorskog prostora $\mathcal{V}$ nad $\mathbb{F}$.
  Definišimo skup $S' = \{L(s) | s \in S\}$.
  Ako bi imali neke skalare $\alpha_1, \ldots,\alpha_n$ tako da
  \[\alpha_1 L(s_1) + \ldots + \alpha_n L(s_n) = 0.\]
  Onda zbog linearnost $L$ imamo
  \[L(\alpha_1 s_1 + \ldots + \alpha_n s_n) = 0.\]
  Kako je $L$ injektivno i $L(0) = 0$ vidimo da je i $\alpha_1 s_1 + \ldots + \alpha_n s_n = 0$.
  Odakle imamo $\alpha_1 = \alpha_2 = \ldots = \alpha_n = 0$.
  Dakle $S'$ je linearno nezavisan skup.
  Ovde takodje vidimo da je $Ker(L) = \{0\}$, što ćemo posle pokazati da implicira $dim(Im(L)) = m$, gde je m dimenzija vektorskog prostora $\mathcal{V}$.


  Za kontradikciju, pretpostavimo da $L$ nije injektivna i da salje linearno nezavisne vektore u linearno nezavisne vektore.
  Onda bi za neke $x_1 \neq x_2$ imali $L(x_1) = L(x_2)$.
  Odatle sledi $L(x_1 - x_2) = 0$.
  Kako je $x_1 - x_2$ linearno nezavisno a $0$ linearno zavisan vektor, imamo kontradikciju.
  Dakle $L$ mora biti injektivna.
\end{proof}

\begin{definition}[Slika]
  \textit{Slika} linearne transformacije iz $\mathcal{V}$ u $\mathcal{W}$ je skup svih izlaznih vektora preslikavanja, ili prostor generisan kolonama matrice koja odgovara linearnom preslikavanju.
  Precizno, 
  \[Im(L) = \{y \in \mathcal{W} | \exists x \in \mathcal{V}, y = L(x)\}.\]
  Dimenzija slike se zove \textit{rang} matrice.
\end{definition}

\begin{definition}[Jezgro]
  \textit{Jezgro} ili nul-prostor linearne transformacije iz $\mathcal{V}$ u $\mathcal{W}$ je skup svih vektora koji se preslikavanju u $0$ vektor.
  Precizno, 
  \[Ker(L) = \{x \in \mathcal{V} | L(x) = 0\}.\]
  Dimenzija jezgra zove se \textit{defekt} matrice.
\end{definition}

\begin{theorem}
  Neka su $\mathcal{V}$ i $\mathcal{W}$ vektorski prostor nad $\mathbb{F}$ i neka je $L$ linearna transformacija iz $\mathcal{V}$ u $\mathcal{W}$.
  Ako je $\mathcal{V}$ konačno-dimenziona onda
  \[dim(Im(L)) + dim(Ker(L)) = dim(\mathcal{V}).\]
\end{theorem}

\begin{proof}
  Glavna ideja dokaza je da ako imamo bazu jezgra $\{x_1, \ldots, x_k\}$, onda postoji baza $\mathcal{V}$ $\{x_1, \ldots, x_n\}$ takva da je $\{L(x_{k + 1}), \ldots, L(x_n)\}$ baza slike.

  Koristimo činjenicu da od svakog linearno nezavisnog skupa možemo da formiramo bazu, kako bi od baze jezgra formirali bazu celog vektorskog prostora $\{x_1, \ldots, x_n\}$.
  Očigledno je da $\{L(x_1), \ldots, L(x_n)\}$ generiše sliku, a kako imamo da $L(x_1) = \ldots = L(x_k) = 0$, vidimo da $\{L(x_{k + 1}, \ldots, x_n)\}$ isto generiše sliku.

  Kako bi videli da su linearno nezavisni uzimimo $\sum_{i = k + 1}^{n} \alpha_i (L(x_i)) = 0$.
  Zbog linearnosti $L$ imamo $L(\sum_{i = k + 1}^{n} \alpha_i x_i) = 0$, odakle vidimo vektor $v = \sum_{i = k + 1}^{n} \alpha_i x_i$ u jezgru.
  Dakle možemo ga sapisati kao $v = \sum_{i = 1}^{k} \beta_i x_i$.
  Odatle imamo $\sum_{i = 1}^{k} \beta_i x_i - \sum_{i = k + 1}^{n} \alpha_i x_i = 0$.
  Kako su $x_1, \ldots, x_n$ linearno nezavisni imamo $\beta_1 = \ldots = \beta_k = \alpha_{k + 1} = \ldots = \alpha_n = 0$.
  Dakle $L(x_{k + 1}), \ldots, L(x_n)$ su linearno nezavisni, i formiraju bazu slike.
\end{proof}

\begin{theorem}
  Neka su $\mathcal{V}$ i $\mathcal{W}$ n-dimenzioni i m-dimenzioni vektorski prostori nad poljem $\mathbb{F}$.
  Neka su $\mathcal{B} = \{x_1, \ldots, x_n\}$ i $\mathcal{B}' = \{y_1, \ldots, y_m\}$ baze tih vektorskih prostora.
  Onda postoji isomorfizam iz prostora svih linearnih preslikavanja $L$ ($L(\mathcal{V}, \mathcal{W})$) u skup svih $m \times n$ matrica $A$ sa unosima iz $\mathbb{F}$ ($M_{m \times n}(\mathbb{F})$).
\end{theorem}

\begin{proof}
  Za dato linearno preslikavanja $L \in L(\mathcal{V}, \mathcal{W})$ formiraćemo njegovu matričnu representaciju na sledeći način.
  Neka se $j$-ta kolona matrice $A$ sastoji od koordinata vektora $L(x_j)$ u bazi $\mathcal{B}'$, ili ti elemetni $A$ su tačno $\alpha_{ij}$ iz $L(x_j) = \sum_{i = 1}^{m} \alpha_{ij} y_i$
  Nazovimo ovu funkciju $f : L(\mathcal{V}, \mathcal{W}) \to M_{m \times n} (\mathbb{F})$.

  Da bi pokazali da je ova funkcija isomorfizam potrebno je da pokažemo da je linearno preslikavanja i da je bijekcija.
  
  Prvo ćemo dokazati da je $f$ linearno preslikavanje.
  Cilj nam je da pokažemo da je $f(k \cdot L_1 + L_2) = k \cdot f(L_1) + f(L_2)$.
  Neka su $A$ i $B$ matrice koje odgovaraju $L_1$ i $L_2$ redom.
  S obzirom da je linearno preslikavanja jedinstveno odredjeno time gde šalje bazne vektore, posmatraćemo kako linearno preslikavanja utiče na neki bazni vektor.
  Dakle interesuje nas $(k \cdot L_1 + L_2)(x_j) = Cx$, gde je $C$ $m \times n$ matrica.
  Koristeći osobine linearnog preslikavanja kao linearnog prostora imamo
  \[(k \dot L_1 + L_2)(x_j) = k \cdot L_1(x_j) + L_2(x_j) = k \cdot \sum_{i = 1}^{m} \alpha_{ij} y_i + \sum_{i = 1}^{m} \beta_{ij} y_i = \sum_{i = 1}^{m} (k \cdot \alpha_{ij} + \beta_{ij}) y_i.\]
  Ovo je tačno sabiranje i množenje skalarom matrica, $C = k \cdot A + B$.
  

  Kako bi videli da je $f$ bijekcija, pokazaćemo da je injekcija i surjekcija.

  Poprilično je očigledno da jezgro preslikavanja $f$ sadrži samo nula transformaciju, jer bi $f(L) = 0$ značilo da se svaki bazni vektor salje u $0$ vektor, odakle se svi vektor salju u $0$ vektor.
  Dakle vidimo da je $f$ injektivna.

  Sa obzirom da je linearna transformacija definisana time gde salje bazne vektore, za svaku $m \times n$ matricu možemo uzeti kolone kao koordinate gde padaju bazni vektori i imamo linearnu transformaciju.

  Dakle naša funkcija $f$ je isomorfizam.
\end{proof}

U algebri dve izomorfne strukture se često smatraju "istim", dakle vidimo da su linearne transformacije i matrice zapravo iste.

\section{Invertibilnost i singularnost matrica}

\begin{definition}[Invertibilnost]
  Matrica $A$ je \textit{invertibilna} ako postoji matrica $A^{-1}$ tako da $AA^{-1} = A^{-1}A = I$.
  Matricu koja nema inverznu matricu nazivamo \textit{singularna}.
\end{definition}

\begin{definition}[Determinanta]
  Neka je $\mathbb{F}$ polje.
  Determinanta je funkcija $det : M_{n \times n} (\mathbb{F}) \to \mathbb{F}$ sa sledećim osobinama
  \begin{enumerate}
    \item Multilinearna je, ako fiksiramo sve sem jedne kolone (ili reda) funkcija je zapravo linearno preslikavanje.
    \item Alternira, ako matrica $A$ imamo dve iste kolone onda $det(A) = 0$.
    \item Slika identitet u $1$, $det(I) = 1$.
  \end{enumerate}

  Moguće je dokazati da je ovakva funkcija jedinstven tako što od ovih svojstava izvedemo sledeću formulu, ponznatu kao Lajbnicova formula.
  \[det(A) = \sum_{\sigma \in S_n} sgn(\sigma) \prod_{i = 1}^{n} a_{\sigma(i), i}.\]
  Gde je $S_n$ simetrična grupa sa $n$ elemenata, $\sigma$ permutacije iz $S_n$ i $sgn$ funkcija koja daje rezlutate $+1$ i $-1$ za parne i neparne permutacije, respektivno.
\end{definition}

Geometrijski (u $\R^n$), determinanta predstavlja faktor sa kojim se menja n-dimenziona zapremina svih figura.
Kao što ćemo videti, ako je determinanta jednaka $0$ to znači da sve površina postaju $0$, dakle gubimo dimenziju.

Direktna posledica alternativnosti je da menjanje dve kolone rezultira u menjanju znaka determinante.

Sledeće su neke od najbitnijih osobina determinante.

\begin{theorem}
  Ako su kolone matrice $A$ linearno zavisne onda je $det(A) = 0$.
\end{theorem}

\begin{proof}
  Ako su kolone matrice $A$ linearno zavisne onda jednu kolonu možemo napisati kao linearnu kombinaciju drugih kolona.
  Fiksirajmo sve ostale kolone $A = \begin{bmatrix} A_1 & \ldots & A_{n - 1} & \sum_{i = 1}^{n - 1} \alpha_i A_i \end{bmatrix}$.
  Zbog multilinearnosti determinante ovo je jednako $\sum_{i = 1}^{n - 1} \begin{bmatrix} A_1 & \ldots & A_{n - 1} & A_i \end{bmatrix}$.
  Kako svaka matrica u ovom zbiru ima dve iste kolone sve determinante moraju biti jednake nuli, pa je i njihov zbir 0.
\end{proof}

\begin{theorem}
  Za bilo koje dve $n \times n$ matrice $A$ i $B$, $det(AB) = det(A)det(B)$.
\end{theorem}

\begin{proof}
  Daćemo samo skicu dokaza

  Fiksirajmo matricu $A$ i posmatrajmo funkciju $D(B) = det(AB)$.
  Dokazujemo da je ova funkcija multilinearna i alternativna.
  Odatle znamo da je naša funkcija zapravo forme $D(B) = \alpha \cdot det(B)$, za neki skalar $\alpha$.
  Stavljajući $B = I$ imamo $\alpha = det(A)$.
\end{proof}

\begin{theorem}
  Matrica $A$ je invertibilna ako i samo ako je $det(A) \neq 0$.
\end{theorem}

\begin{proof}
  Pretpostavimo da je $A$ invertibilna.
  Onda postoji matrica $A^{-1}$ takva da je $AA^{-1} = I$.
  Koristeći prethodnu teoremu imamo $1 = det(I) = det(AA^{-1}) = det(A)det(A^{-1})$.
  Odavde očigledno sledi $det(A) \neq 0$.


  Uzmimo sad da je $det(A) \neq 0$.
  Odavde direktno vidimo da su kolone $A$ linearno nezavisne.
  Dakle, kolone $A$ formiraju bazu, i $A$ predstavlja linearnu transformaciju koja slika bazu u bazu.
  To znači da postoji inverzna linearna transformacija, čiji je matrični prikaz zapravo $A^{-1}$.
\end{proof}

Odavde možemo da vidimo da ako je $AB$ invertibilna, direktno sledi da su i $A$ i $B$ invertibilne, jer u suprotnom bi imali da $x = 0 \cdot y$, gde $x \neq 0$.
Takodje imamo $det(AB) = det(A)det(B) = det(B)det(A) = det(BA)$, tako da je i matrica $BA$ invertibilna.

Za poslednju osobinu nam prvo treba definicija transponovanja matrice.

\begin{definition}[Transponovanje matrice]
  Neka je $A$ $m \times n$ matrica, onda dobijamo \textit{transponovanu matricu} $A^T$ kada zamenimo mesta redovima i kolonama matrice $A$.
\end{definition}

\begin{theorem}
  Za bilo koju matricu $A$ imamo $det(A) = det(A^T)$.
\end{theorem}

\begin{proof}
  Ove se dokazuje sa analizom Lajbnicove formule.
\end{proof}

Već smo pokazali da je linearno preslikavanja iz $\mathcal{V}$ u $\mathcal{W}$ injektivno ako i samo ako je $Ker(L) = \{0\}$.
Linearno preslikavanje je surjektivno kada je $Im(L) = dim(\mathcal{W})$.
Bilo koja funkcija je bijektivna ako je injektivna i surjektivna.

Primetimo da za linearno preslikavanja sa kvadratnom matricom su sve tri osobine ekvivalentne zbog teoreme of rangu i defektu.
Dakle imamo sledeću teoremu.

\begin{theorem}
  Za kvadratnu $n \times n$ matricu $A$ sledeće je ekvivalentno
  \begin{enumerate}
    \item Matrica $A$ je bijektivna;
    \item $det(A) \neq 0$;
    \item $rang(A) = n$;
    \item $Ker(A) = \{0\}$.
  \end{enumerate}
\end{theorem}

\section{Sopstvene vrednosti i karakteristični polinom}

\begin{definition}[Sopstveni vektori i sopstvene vrednosti]
  \textit{Sopstveni vektor} matrice $A$ je nenula vektor $x$ koji ne menja pravac pod delovanjem matrice, tj., $Ax = \lambda x$, za neki skalar $\lambda$ koji nazivamo \textit{sopstvenom vrednosti}.
\end{definition}

Sopstvene vektore i sopstvene vrednosti dobijamo posmatrajući karakterističnu matricu $(A - \lambda I)$ koju lako dobijamo iz definicije.
Da bi postojala netrivijalna rešenja potrebno je da ta matrica singularna.

\begin{definition}[Karakteristični polinom]
  \textit{Karakteristični polinom} matrice $A$ je $p(x) = det(A - x I)$.
  Nule karakterističnog polinoma su sve sopstvene vrednosti matrice $A$.
\end{definition}

Odavde vidimo da za $n \times n$ matricu $A$ nad algebarski zatvorenom polju $\mathbb{F}$ imamo tačno $n$ sopstvenih vrednosti uključujući višestrukosti.

\begin{definition}[Spektar]
  Skup svih sopstvenih vrednosti matrice $A$ naziva se \textit{spektar} matrice.
\end{definition}

\begin{definition}[Algebarska višestrukost]
  Neka je $\lambda$ sopstvena vrednost matrice $A$.
  \textit{Algebarska višestrukost} od $\lambda$ je broj ponavljanja $\lambda$ kao korena karakterističnog polinoma.
  Na primer, ako je $p(x) = (x - \lambda_1)^{\alpha_1} \cdot \ldots \cdot (x - \lambda_n)^{\alpha_n}$ onda je algebarska višestrukost od $\lambda_i$ zapravo skalar $\alpha_i$.
\end{definition}

\begin{definition}[Geometrijska višestrukost]
  \textit{Geometrijska višestrukost} je definisana kao dimenzija sopstvenog prostora.
  Sopstveni prostor definišemo kao jezgro matrice $A - \lambda I$.
\end{definition}

Geometrijski ova definicija predstavlja broj pravaca koji se samo izdužuju (ili skupljaju).

\begin{theorem}
  Geometrijska višestrukost je uvek manja ili jednaka algebarskoj višestrukosti.
\end{theorem}

\begin{theorem}
  Zbir svih sopstvenih vrednosti jednak je tragu (zbir glavne dijagonale) matrice $A$.
  Proizvod sopstvenih vrednosti jednak je determinanti $A$.
\end{theorem}

\begin{proof}
  Posmatrajmo karakteristični polinom $p(x) = det(A - x I)$.
  Iz Vietovih formula znamo da je $\sum \lambda_i = -\frac{c_{n - 1}}{c_n}$ i $\prod \lambda_i = (-1)^n \cdot \frac{c_0}{c_n}$, gde su $c_i$ koeficijenti ispred člana $x^i$.

  Primetimo da kako bi dobili $x^n$ potrebno je da pomnožimo dijagonalu $A - x I$, tj., imamo $c_n x^n = \prod (\alpha_{ii} - x)$.
  Odatle je očigledno $c_n = (-1)^n$.

  Kako bi dobili $c_{n - 1}$ potrebno je da zamenimo jedno $x$ sa $\alpha_{ii}$ u našem proizvodu.
  To možemo uraditi na $n$ načina i dobijamo $c_{n - 1} x^{n - 1} = \sum_{i = 1}^{n} \alpha_{ii} (-x^{n - 1}) = x^{n - 1} \sum_{i = 1}^{n} (-1)^{n - 1} \alpha_{ii} = Tr(A) (-1)^{n - 1} x^{n - 1}$ gde je $Tr(A)$ trag od $A$.
  Dakle, $c_{n - 1} = Tr(A) (-1)^{n - 1}$.

  Konačno, da bismo našli $c_0$ potrebno je samo da pogledamo $p(0) = det(A - 0 \cdot I) = det(A)$.

  Kada ubacimo ove vrednosti u Vietove formule dobijamo
  \[\sum \lambda_i = -\frac{Tr(A) (-1)^{n - 1}}{(-1)^n} = Tr(A).\]
  \[\prod \lambda = (-1)^n \frac{det(A)}{(-1)^n} = det(A).\]
\end{proof}

\begin{theorem}
  Matrica $A$ i transponovana matrica $A^T$ imaju iste sopstvene vrednosti sa istim algebarskim višestrukostima.
\end{theorem}

\begin{proof}
  Dokaz trivijalno sledi iz osobina transponovanja, specifično $det(A) = det(A^T)$, $(k \cdot A + B)^T = k \cdot A^T + B^T$ i $I = I^T$.
\end{proof}

\begin{definition}
  Sopstvene vektore matrice $A$ takodje nazivamo desnim sosptvenim vektorima, dok su sosptveni vektori matrice $A^T$ levi sosptveni vektori.
\end{definition}

\begin{theorem}[Diagonalna matrica]
  Matrica $A$ je slična dijagonalnoj matrici ako i samo ako su algebarska i geometrijska višestrukost jednake za sve sosptvene vrednosti.
\end{theorem}

\begin{corollary}
  Ako $n \times n$ matrica $A$ ima $n$ sosptvenih vrednosti onda je slična dijagonalnoj matrici.
\end{corollary}

Dijagonalne matrice su generalno matrice koje se veoma lepo ponašaju i uprošćavaju mnoge stvari.

\end{document}
