\documentclass{article}

% ===================================================================
% PREAMBLE FOR MATHEMATICS EXERCISE NOTEBOOK
% ===================================================================
% This file contains all the packages and custom commands for the document.
% Keeping it separate from main.tex makes the project cleaner.


% -------------------------------------------------------------------
% DOCUMENT & ENCODING
% -------------------------------------------------------------------
\usepackage[utf8]{inputenc} % Allows you to type characters like á, ö, etc. directly
\usepackage[T1]{fontenc}    % Specifies font encoding, improves font rendering and hyphenation


% -------------------------------------------------------------------
% PAGE LAYOUT & GEOMETRY
% -------------------------------------------------------------------
\usepackage{geometry}
\geometry{
  a4paper,         % Or letterpaper, etc.
  total={170mm,257mm},
  left=20mm,
  top=20mm,
}


% -------------------------------------------------------------------
% CORE MATH PACKAGES (AMS - American Mathematical Society)
% -------------------------------------------------------------------
\usepackage{amsmath}    % The premier package for typesetting math equations
\usepackage{amssymb}    % Provides lots of extra math symbols (like \mathbb)
\usepackage{amsthm}     % Comprehensive theorem-like environments
\usepackage{mathtools}  % An extension of amsmath, provides more tools and fixes


% -------------------------------------------------------------------
% UTILITY & FORMATTING PACKAGES
% -------------------------------------------------------------------
\usepackage{graphicx}      % For including images (\includegraphics)
\usepackage{xcolor}        % For defining and using colors
\usepackage{booktabs}      % For creating beautiful, professional-looking tables (\toprule, \midrule, \bottomrule)
\usepackage{siunitx}       % For typesetting numbers and units consistently
\usepackage{enumitem}      % Provides more control over list environments (itemize, enumerate)


% -------------------------------------------------------------------
% HYPERLINKS & CROSS-REFERENCING
% -------------------------------------------------------------------
\usepackage{hyperref}
\hypersetup{
    colorlinks=true,       % false: boxed links; true: colored links
    linkcolor=teal,        % color of internal links (e.g., sections)
    citecolor=green,       % color of links to bibliography
    filecolor=magenta,     % color of file links
    urlcolor=blue          % color of external links
}


% -------------------------------------------------------------------
% THEOREM & DEFINITION ENVIRONMENTS
% -------------------------------------------------------------------
% This section sets up consistent numbering and styling for problems,
% definitions, theorems, etc.

\theoremstyle{definition} % Use a style that is less flashy than the default "plain" style
\newtheorem{definition}{Definition}[section] % Number problems as Problem X.Y (where X is the section number)
\newtheorem{example}[definition]{Example}

\theoremstyle{plain}
\newtheorem{theorem}[definition]{Theorem}     % Share the same counter as 'problem'
\newtheorem{lemma}[definition]{Lemma}
\newtheorem{corollary}[definition]{Corollary}

\theoremstyle{remark}
\newtheorem{remark}[definition]{Remark}

\renewcommand{\proofname}{Proof}
% -------------------------------------------------------------------
% CUSTOM MATH COMMANDS (MACROS)
% -------------------------------------------------------------------
% Define shortcuts for commonly used mathematical notation to save
% time and ensure consistency.

% Sets of numbers
\newcommand{\R}{\mathbb{R}} % Real numbers
\newcommand{\C}{\mathbb{C}} % Complex numbers
\newcommand{\N}{\mathbb{N}} % Natural numbers
\newcommand{\Z}{\mathbb{Z}} % Integers
\newcommand{\Q}{\mathbb{Q}} % Rational numbers

% Calculus operators
\newcommand{\dd}{\, \mathrm{d}} % For integrals, e.g., \int f(x)\dd x
\newcommand{\pdv}[2]{\frac{\partial #1}{\partial #2}} % Partial derivative
\newcommand{\dv}[2]{\frac{\mathrm{d} #1}{\mathrm{d} #2}} % Full derivative

% Linear Algebra
\DeclareMathOperator{\Tr}{Tr} % Trace of a matrix
\newcommand{\T}{\mathsf{T}}   % Transpose, e.g., A^\T

% Probability & Statistics
\DeclareMathOperator{\E}{\mathbb{E}} % Expectation
\DeclareMathOperator{\Var}{Var} % Variance
\DeclareMathOperator{\Cov}{Cov} % Covariance


% ===================================================================
% END OF PREAMBLE
% ===================================================================


\title{Yang-Baxter-like matrix equations}
\author{Mihailo Đurić}
\date{\today}

\begin{document}

\maketitle
\tableofcontents
\newpage

\section{Preliminary results}

\begin{theorem} \label{diagonalizable}
  A linear operator $A$ on a vector space $V$ of dimension $n$ over the field $\C$ is diagonalizable if and only if the minimal polynomial of $A$ has no repeated roots.
\end{theorem}

\begin{theorem} \label{minimalpoly}
  Every annihilating polynomial is the multiple of the minimal polynomial of an operator $A$.
\end{theorem}

\begin{theorem} \label{solutions}
  Suppose that $A$ is an $n \times n$ complex diagonalizable matrix with three distinct eigenvalues $0, \lambda$ and $\mu$ with $\lambda \mu \neq 0$ and $\lambda^2 - \lambda \mu + \mu^2 \neq 0$.
  Then all solutions of the Yang-Baxter-like matrix equation $A X A = X A X$ have the form
  \[X = S \begin{bmatrix} P & 0 & 0\\ 0 & Q & 0\\ 0 & 0 & I_{n - m} \end{bmatrix} \left[ \begin{array}{ccc|ccc|c} \widehat{\lambda} I_r & 0 & 0 & F & 0 & 0 & 0 \\ 0 & \lambda I_v & 0 & 0 & 0 & 0 & 0 \\ 0 & 0 & 0_{k - r - v} & 0 & 0 & 0 & C_3 \\ \hline G & 0 & 0 & \widehat{\mu} I_r & 0 & 0 & 0 \\ 0 & 0 & 0 & 0 & \mu I_t & 0 & 0 \\ 0 & 0 & 0 & 0 & 0 & 0_{m - k - r - t} & C_6 \\ \hline 0 & 0 & D_3 & 0 & 0 & D_6 & W \\ \end{array} \right] \begin{bmatrix} P^{-1} & 0 & 0\\ 0 & Q^{-1} & 0\\ 0 & 0 & I_{n - m} \end{bmatrix} S^{-1},\]
  in which, $P \in \C^{k \times k}, Q \in \C^{(m - k) \times (m - k)}$ are any invertible matrices, $0 \le r \le \min \{k, m - k\}, 0 \le v \l k - r, 0 \le t \le m - k - r, \widehat{\lambda} = \frac{\mu^2}{\mu - \lambda}, \widehat{\mu} = \frac{\lambda^2}{\lambda - \mu}$, $F$ is an arbitrary $r \times r$ invertible matrix, $G = \frac{- \lambda \mu (\lambda^2 - \lambda \mu + \mu^2)}{(\lambda - \mu)^2} F^{-1}$, $C_3 \in \C^{(k - r - v) \times (n - m)}$, $C_6 \in \C^{(m - k - r - t) \times (n - m)}$, $D_3 \in \C^{(n - m) \times (k - r - v)}$, $D_6 \in \C^{(n - m) \times (m - k - r - t)}$, $\lambda D_3 C_3 = - \mu D_6 C_6$, and $W$ is an arbitrary $(n - m) \times (n - m)$ matrix.
\end{theorem}

\begin{theorem} \label{similarity}
  If $A = Q^{-1} J Q$ and $Y$ is a solution of the equation $J Y J = Y J Y$, then $X = Q^{-1} Y Q$ is a solution to the equation $A X A = X A X$.
\end{theorem}

\begin{theorem}[Core-Nilpotent Decomposition] \label{core-nilpotent}
Let $A$ be a square matrix of size $n \times n$ with entries in the complex numbers. Let $k = \mathrm{ind}(A)$ be the index of $A$, and let $r = \mathrm{rank}(A^k)$. Then there exists a non-singular matrix $Q$ such that
\[
Q^{-1}AQ = 
\begin{bmatrix}
    L & 0 \\
    0 & C
\end{bmatrix}
\]
where $C$ is an $r \times r$ invertible matrix, and $L$ is an $(n-r) \times (n-r)$ nilpotent matrix of index $k$.
\end{theorem}

\begin{theorem} \label{invertible}
  Let $A$ be a regular matrix then all nonzero commuting solutions to $A X A = X A X$ are provided via the formula
  \[X_c = \frac{1}{2} \bigl( A + (A^2)^{1/2} \bigl),\]
  where $(A^2)^{1/2}$ is any square root of the matrix $A^2$.
\end{theorem}

\begin{theorem} \label{Sylverster-equation}
  The equation $AX - XB = C$ has a unique solution $X$ if and only if $\sigma (A) \cap \sigma (B) = \emptyset$
\end{theorem}

\begin{theorem} \label{generating-solutions}
  For a given matrix $A \in M_n (\C)$, let $X_0 \in S$ and let $f, g$ be well defined functions in $\sigma (A)$ such that
  \[g(A) A f(A) = A.\]
  Then $f(A) X_0 g(A) \in S$.
\end{theorem}

\section{All solutions for $A^3 = -A$}

\begin{theorem} \label{periodic-diagonal}
  For any $n \in \N$ and $t \in \C$, the matrix $A$ satisfying $A^n = t A$ is diagonalizable.
\end{theorem}

\begin{proof}
  Consider the polynomial $p(x) = x^n - t x$.
  It is clear that any $A$ such that $A^n = t A$ annihilates this polynomial, thus by Theorem \ref{minimalpoly}, it is a multiple of the minimal polynomial.
  Our polynomial has $n$ distinct roots, $x = 0$ and $x^{n - 1} = t$.
  Thus the roots of the minimal polynomial must also be distinct, i.e., have multiplicity 1.
  Applying Theorem \ref{diagonalizable} we get that $A$ is diagonalizable.
\end{proof}

\begin{theorem}
  If $A^3 = -A$, assume that the rank of $A$ is $m$ and the multiplicity of eigenvalue $i$ is $k$.
  Then all solutions of the Yang-Baxter-like matrix equation $A X A = X A X$ have the form
  \[X = S \begin{bmatrix} P & 0 & 0\\ 0 & Q & 0\\ 0 & 0 & I_{n - m} \end{bmatrix} \left[ \begin{array}{ccc|ccc|c} - \frac{i}{2} I_r & 0 & 0 & F & 0 & 0 & 0 \\ 0 & i I_v & 0 & 0 & 0 & 0 & 0 \\ 0 & 0 & 0_{k - r - v} & 0 & 0 & 0 & C_3 \\ \hline - \frac{3}{4} F^{-1} & 0 & 0 & \frac{i}{2} I_r & 0 & 0 & 0 \\ 0 & 0 & 0 & 0 & - i I_t & 0 & 0 \\ 0 & 0 & 0 & 0 & 0 & 0_{m - k - r - t} & C_6 \\ \hline 0 & 0 & D_3 & 0 & 0 & D_6 & W \\ \end{array} \right] \begin{bmatrix} P^{-1} & 0 & 0\\ 0 & Q^{-1} & 0\\ 0 & 0 & I_{n - m} \end{bmatrix} S^{-1},\]
  in which, $P \in \C^{k \times k}, Q \in \C^{(m - k) \times (m - k)}$ are any invertible matrices, $0 \le r \le \min \{k, m - k\}, 0 \le v \l k - r, 0 \le t \le m - k - r$, $F$ is an arbitrary $r \times r$ invertible matrix, $C_3 \in \C^{(k - r - v) \times (n - m)}$, $C_6 \in \C^{(m - k - r - t) \times (n - m)}$, $D_3 \in \C^{(n - m) \times (k - r - v)}$, $D_6 \in \C^{(n - m) \times (m - k - r - t)}$, $i D_3 C_3 = i D_6 C_6$, and $W$ is an arbitrary $(n - m) \times (n - m)$ matrix.
\end{theorem}

\begin{proof}
  By Theorem \ref{periodic-diagonal} we know that $A$ is diagonalizable and that its eigenvalues are in the set $\{0, i, - i\}$.
  We easily check that $i^2 - i (- i) + (- i)^2 = -3 \neq 0$.
  Applying Theorem \ref{solutions} we get our result.
\end{proof}

\begin{remark}
This trivially generalizes for the case of $A^3 = t A$ for any $t \in \C$.
\end{remark}

\section{Some solutions for $A^n = A^2$}
\begin{theorem} \label{singular}
  If $A$ is an invertible matrix, then all nontrivial commuting solutions to Yang-Baxter-like matrix equations are singular.
\end{theorem}

\begin{proof}
  It is easy to show that if we assume $X$ to be a commuting solution, the the Yang-Baxter-like matrix equation is equivalent to the equation $AX = X^2 = XA$.
  If $X$ is invertible, it follows that $AX = X^2$ implies $A = X$.
  By the contrapositive we know that if $X \neq A$ is a commuting solution to the Yang-Baxter-like matrix equation, then $X$ must be singular.
\end{proof}

\begin{algorithm}
  Here we present a way of generating families of solutions of the Yang-Baxter-like matrix equations for the case that $A^n = A^2$.

  We begin by decomposing our matrix $A$ into a block diagonal form using Theorem \ref{core-nilpotent}.
  Any solution found for this new matrix will yield a solution for $A$ by Theorem \ref{similarity}, thus we will simply refer to our decomposed matrix as $A$.
  Notice that since the minimal polynomial of $A$ divides $m_A(x) = x^2 (x^{n - 2} - 1)$ our nilpotent matrix, $A_1$ will have a nilpotency index 2, and our invertible matrix $A_2$ will be a diagonalizable matrix.
  Since $A_4$ is diagonalizable, we can simply use Theorem \ref{similarity} and assume that it is diagonal.
  We put
  \[X = \begin{bmatrix} X_1 & X_2\\ X_3 & X_4 \end{bmatrix}.\]
  Then, multiplying out $A X A = X A X$ we get the following system of equations:
  \[\begin{cases}
    A_1 X_1 A_1 = X_1 A_1 X_1 + X_2 A_2 X_3\\
    A_1 X_2 A_2 = X_1 A_1 X_2 + X_2 A_2 X_4\\
    A_2 X_3 A_1 = X_3 A_1 X_1 + X_4 A_2 X_3\\
    A_2 X_4 A_2 = X_3 A_1 X_2 + X_4 A_2 X_4
  \end{cases}\]
  From here we consider multiple cases.

  \begin{case}[$X_2 = X_3 = 0$]
    Our system of equation reduces to two Yang-Baxter-like matrix equations, one for the nilpotent matrix $A_1$ and one for the invertible matrix $A_2$.
    All of the solutions for the nilpotent matrix with a nilpotency index 2 have been found in the paper D. Zhou and J. Ding, \textit{All solutions of the Yang-Baxter-like matrix equation for nilpotent matrices of index two}.
    Using Theorem \ref{invertible} we can find all the commuting solutions for $A_2$.
    It is easy to check that all solutions for this case will be commutative.
  \end{case}

  \begin{case}[$X_3 = 0, X_2 \neq 0$]
    Here our system simplifies to
    \[\begin{cases}
      A_1 X_1 A_1 = X_1 A_1 X_1\\
      A_1 X_2 A_2 = X_1 A_1 X_2 + X_2 A_2 X_4\\
      A_2 X_4 A_2 = X_4 A_2 X_4
    \end{cases}\]
    Again, we can use Theorem \ref{invertible} to find all commuting $X_4$.
    We will choose a singular $X_4$ which we know exists by Theorem \ref{singular}.
    Since $A_1$ is singular, we know that there exist infinitely many matrices such that $X_1 A_1 = 0$, and they all clearly provide a solution to the first equation.
    If we pick such an $X_1$, the second equation reduces to the following:
    \[A_1 X_2 - X_2 A_2 X_4 A_2^{-1} = 0.\]
    This is a Sylverster equation with a trivial solution $X_2 = 0$.
    Since $X_4$ is singular so is $A_2 X_4 A_2^{-1}$ as eigenvalues are invariant under change of basis.
    Clearly $A_1$ is singular since it is nilpotent.
    Thus their spectra have a nonempty intersection, as $0$ is an eigenvalue of both matrices since they are singular.
    Hence, by Theorem \ref{Sylverster-equation} it follows that $X_2 = 0$ is not a unique solution.
    It is possible to construct all of the solutions for this Sylverster equation, giving us another family of solutions for the Yang-Baxter-like matrix equation.
    It is easy to check that as long as $X_2 \neq 0$ these solutions are all non commuting.
  \end{case}

  \begin{case}[$X_2 = 0, X_3 \neq 0$]
    This case is analogous to the previous one with a few changes.
  \end{case}

  Since we have found non commuting solutions to the Yang-Baxter-like matrix equation we can use Theorem \ref{generating-solutions} to perhaps find more solutions
  Note that by definition $f(A)$ and $g(A)$ must be block diagonal matrices.
  Thus we can write
  \[f(A) = \begin{bmatrix} F_1 (A) & 0\\ 0 & F_2 (A) \end{bmatrix}, g(A) = \begin{bmatrix} G_1 (A) & 0\\ 0 & G_2 (A) \end{bmatrix}.\]
  Since the solutions we have found are block upper triangular it is easy to see that
  \[f(A) X g(A) = \begin{bmatrix} F_1 (A) X_1 G_1 (A) & F_1 (A) X_2 G_2 (A)\\ 0 & F_2 (A) X_4 G_2 (A) \end{bmatrix}.\]
  Thus we certainly cannot generate all solutions starting from this our non commuting solution.

  \begin{case}[$X_2 \neq 0, X_3 \neq 0$]
    The solution for this would require solving the full system which is currently too difficult.
    However, such solutions can exist in general, take
    \[A = \begin{bmatrix} 0 & 1 & 0 & 0\\ 0 & 0 & 0 & 0\\ 0 & 0 & 1 & 0\\ 0 & 0 & 0 & -1 \end{bmatrix}, X = \begin{bmatrix} 0 & 0 & 0 & 2\\ 0 & 0 & 0 & 0\\ 0 & 3 & 0 & 0\\ 0 & 0 & 0 & 0 \end{bmatrix}.\]
    It is simple to check that $A^4 = A^2$ and that $A X A = X A X$.
  \end{case}
\end{algorithm}

\begin{remark}
  This methodology for generating solutions can be easily generalized to the arbitrary case of a singular non-nilpotent matrix.
  The noncommuting solutions remain the same, while for commuting solutions we become limited to only the zero divisors of $A_1$, since the general nilpotent case is still open.
\end{remark}

\section{Diagonal Matrices}
If we have a diagonal matrix $A$, it makes sense to build on our previous work by using the core-nilpotent decomposition.
Here the matrix $A$ decomposes into a $0$ matrix and an invertible, diagonal matrix $A_2$.
Considering $X$ as a $2 \times 2$ block matrix, our previous system of equation simplifies to
\[\begin{cases}
  X_2 A_2 X_3 = 0\\
  X_2 A_2 X_4 = 0\\
  X_4 A_2 X_3 = 0\\
  A_2 X_4 A_2 = X_4 A_2 X_4
\end{cases}\]
For the solutions to be commuting we can again find the following system
\[\begin{cases}
  X_2 A_2 = 0\\
  A_2 X_3 = 0\\
  A_2 X_4 = X_4 A_2
\end{cases}\]
Since $A_2$ is invertible, it follows that it is not a zero-divisor, thus we know that $X_2 = X_3 = 0$.
All commuting solutions for $X_4$ can be given by the Theorem \ref{invertible}.
Thus all commuting solutions for a diagonal matrix are given by an arbitrary $X_1$, $X_2 = X_3 = 0$ and a commuting solution to $A_2 X_4 A_2 = X_4 A_2 X_4$.

The commuting solution were already known, so we are interested in any solution not satisfying the previous equations.
These solutions either have a non-commuting solution for $X_4$ or a non-zero $X_2$ or $X_3$.

We now present a procedure to construct the complete set of solutions to this system.
We proceed under the assumption that the solution set for the fourth equation, the Yang-Baxter-like equation $A_2 X_4 A_2 = X_4 A_2 X_4$, is known.
This set, which is the focus of other parts of this research, will be denoted by $S_4$.
The following proposition provides a constructive algorithm that generates all solutions for the full block matrix $X$.

\begin{theorem} \label{prop:constructive_solution}
Let $A = \text{diag}(0, A_2)$ be a block diagonal matrix where the zero block is of size $(n - m) \times (n - m)$ and $A_2$ is an $m \times m$ invertible, diagonal matrix.
  Let $X = \begin{bmatrix} X_1 & X_2\\ X_3 & X_4 \end{bmatrix}$ be a block matrix of compatible dimensions.
  The matrix $X$ is a solution to $AXA = XAX$ if and only if its constituent blocks are determined by the following procedure:
  \begin{enumerate}
    \item Choose an arbitrary matrix $X_4^* \in S_4$.
    \item Let $N(X_4^*)$ denote the null space of the chosen $X_4^*$.
      Construct the matrix $X_3$ such that its column space, $C(X_3)$, is a subspace of $A_2^{-1}(N(X_4^*))$.
    \item Based on the chosen $X_3$ and $X_4^*$, define the subspace $S = \text{span}(C(A_2 X_3) \cup C(A_2 X_4^*))$.
      Construct the matrix $X_2$ such that its null space, $N(X_2)$, contains the subspace $S$.
    \item Choose $X_1$ to be any arbitrary $(n - m) \times (n - m)$ matrix.
  \end{enumerate}
\end{theorem}

\begin{proof}
  The equation $AXA = XAX$ for the given block matrices simplifies to the system:
  \begin{align}
    X_2 A_2 X_3 &= 0 \label{eq:sys1}\\
    X_2 A_2 X_4 &= 0 \label{eq:sys2}\\
    X_4 A_2 X_3 &= 0 \label{eq:sys3}\\
    A_2 X_4 A_2 &= X_4 A_2 X_4 \label{eq:sys4}
  \end{align}
  We prove that the procedure in Theorem \ref{prop:constructive_solution} is both necessary and sufficient to define the complete solution set for this system.

  \textbf{(Sufficiency)}
  Assume a matrix $X$ is constructed according to the procedure.
  By Step 1, $X_4^*$ is in $S_4$, so equation \eqref{eq:sys4} is satisfied by definition.

  By Step 2, $C(X_3) \subset A_2^{-1} (N(X_4^*))$.
  This implies $C(A_2 X_3) \subset N(X_4^*)$, which is equivalent to the statement $X_4^* A_2 X_3 = 0$.
  Thus, equation \eqref{eq:sys3} is satisfied.

  By Step 3, $N(X_2)$ contains the subspace $S = \text{span}(C(A_2 X_3) \cup C(A_2 X_4^*))$.
  Since $C(A_2 X_3) \subset S \subset N(X_2)$, it follows that $X_2 A_2 X_3 = 0$.
  Thus, equation \eqref{eq:sys1} is satisfied.
  Similarly, since $C(A_2 X_4^*) \subset S \subset N(X_2)$, it follows that \eqref{eq:sys2} is satisfied.

  The matrix $X_1$ does not appear in any of the four equation, so its choice is unconstrained.

  Therefore, any matrix $X$ constructed via this procedure is a valid solution.

  \textbf{(Necessity)}
  Assume $(X_1, X_2, X_3, X_4)$ is a valid solution to the system \eqref{eq:sys1}-\eqref{eq:sys4}.
  We show that it must conform to the procedure.
  \begin{enumerate}
    \item Equation \eqref{eq:sys4} requires that $X_4$ must be a solution to the Yang-Baxter-like matrix equation for $A_2$.
      Therefore $X_4 \in S_4$, satisfying Step 1.
    \item From equation \eqref{eq:sys3}, $X_4 A_2 X_3 = 0$, we deduce that $C(A_2 X_3) \subset N(X_4)$.
      Since $A_2$ is invertible, this implies $C(X_3) \subset A_2^{-1} (N(X_4))$, satisfying the condition of Step 2.
    \item Equations \eqref{eq:sys1} and \eqref{eq:sys2} require that $X_2 A_2 X_3 = 0$ and $X_2 A_2 X_4 = 0$.
      This means the null space of $X_2$ must contain every column of $A_2 X_3$ and every column of $A_2 X_4$.
      Therefore, $N(X_2)$ must contain the span of their union, $S = \text{span} (C(A_2 X_3) \cup C(A_2 X_4)$.
      This is precisely the condition of Step 3.
    \item Since the matrix $X_1$ is not involved in any of the four equations, it is completely unconstrained and can be chosen arbitrarily.
      This conforms with Step 4.
  \end{enumerate}

  Thus, any solution to the system must be constructable via the stated procedure.
\end{proof}
%
% This procedure already allows us to generate some solutions, as the commuting $X_4$ are known.
% To find all solution, we require the entire set $S_4$.
% The general case seems too difficult currently, so we try to find all solutions for an invertible, diagonal matrix with three distinct eigenvalues.
% This means that $A$ is of the form $A = \begin{bmatrix} \lambda_1 I & 0 & 0\\ 0 & \lambda_2 I & 0\\ 0 & 0 & \lambda_3 I \end{bmatrix}$.
% If we assume that $X = \begin{bmatrix} X_{11} & X_{12} & X_{13}\\ X_{21} & X_{22} & X_{23}\\ X_{31} & X_{32} & X_{33} \end{bmatrix}$ with suitable sized blocks, by multiplying everything out we are left with the system of equations of the form
% \[\lambda_i \lambda_j X_{ij} = \sum_{k = 1}^{3} \lambda_k X_{ik} X_{kj}\]
% for some $i, j \in \{1, 2, 3\}$.
% This is similar to the system we would get in the case for more eigenvalues, except with block matrices instead of scalar entries.
% All equations with $i \neq j$ are actually Sylverster-equations.
% For example, consider the equation
% \[\lambda_1 \lambda_2 X_{12} = \lambda_1 X_{11} X_{12} + \lambda_2 X_{12} X_{22} + \lambda_3 X_{13} X_{32}.\]
% We can write this as a Sylverster-equation with an unknown $X_{12}$ in the following form
% \[(\lambda_1 X_{11} - \lambda_1 \lambda_2 I) X_{12} + X_{12} (\lambda_2 X_{22}) = -\lambda_3 X_{13} X_{32}.\]
% This in conjunction with the cases when $i = j$ still seem to general to solve.
% The best result I've gotten is placing $X_{12} = X_{13} = X_{32} = 0$.
% In this case we can find $X_{11}, X_{22}$ and $X_{33}$ from the equations $\lambda_i X_{ii} = X_{ii}^2$.
% It is clear that the eigenvalues of $X_{ii}$ here are a subset of the set $\{\lambda_i, 0\}$ and that the matrices $X_{ii}$ are diagonalizable (consider the minimal polynomial).
% We also have the equation $\lambda_2 \lambda_1 X_{21} = \lambda_1 X_{21} X_{11} + \lambda_2 X_{22} X_{21}$.
% This can be rewritten as a Sylverster-equation of the form $\lambda_2 (X_{22} - \lambda_1 I) X_{21} + X_{21} \lambda_1 X_{11} = 0$.
% We can find that the eigenvalues of $\lambda_2 (X_{22} - \lambda_1 I)$ are a subset of $\{\lambda_2^2 - \lambda_1 \lambda_2, -\lambda_1 \lambda_2\}$.
% Similarly, we find that the eigenvalues of $\lambda_1 X_{11}$ are a subset of $\{\lambda_1^2, 0\}$.
% Since $\lambda_1 \neq \lambda_2$, $\lambda_1 \neq 0$ and $\lambda_2 \neq 0$ it follows that $0$ is not a common eigenvalue.
% If $X_{21} \neq 0$ then either $\lambda_1 = -\lambda_2$ or $\lambda_1^2 + \lambda_1 \lambda_2 - \lambda_2^2 = 0$.
% If either of those two holds, the Sylverster-equation has non-trivial solutions and can be solved since it is homogenous.
% Similarly, we get the equations $\lambda_2 \lambda_3 X_{23} = \lambda_2 X_{22} X_{23} + \lambda_3 X_{23} X_{33}$ and $\lambda_3 \lambda_1 X_{31} = \lambda_1 X_{31} X_{11} + \lambda_3 X_{33} X_{31}$.
% These are solved in an analogous way to above, having non-trivial solutions if $\lambda_2 = - \lambda_3$ or $\lambda_2^2 + \lambda_2 \lambda_3 - \lambda_3^2 = 0$ and $\lambda_3 = - \lambda_1$ or $\lambda_3^2 + \lambda_3 \lambda_1 - \lambda_1^2 = 0$

This procedure already allows us to generate some solutions, as the commuting solutions for $X_4$ are known. To find all solutions, we require the entire set $\mathcal{S}_4$.

The general case for an arbitrary $A_2$ seems too difficult at present. We therefore restrict our analysis to a specific, illustrative case: an invertible, diagonal matrix $A_2$ with three distinct, non-zero eigenvalues. This means $A_2$ is of the form $A_2 = \text{diag}(\lambda_1 I, \lambda_2 I, \lambda_3 I)$. We partition a potential solution $X_4$ conformally with $A_2$ as 
\[
X_4 = 
\begin{bmatrix}
    X_{11} & X_{12} & X_{13} \\
    X_{21} & X_{22} & X_{23} \\
    X_{31} & X_{32} & X_{33}
\end{bmatrix}.
\]
Multiplying out the equation $A_2 X_4 A_2 = X_4 A_2 X_4$ yields a system of nine block matrix equations of the form:
\[
\lambda_i \lambda_j X_{ij} = \sum_{k = 1}^{3} \lambda_k X_{ik} X_{kj} \quad \text{for } i, j \in \{1, 2, 3\}.
\]

The equations where $i \neq j$ can be framed as Sylvester equations, but the full system remains coupled and difficult to solve. To make progress, we investigate a special family of solutions by imposing a structural constraint on $X_4$.

\subsection*{A Family of Solutions with a Specific Block Structure}

Let us search for solutions where $X_{12} = X_{13} = X_{32} = 0$. This simplifying assumption decouples the system considerably.

The equations for the diagonal blocks reduce to $\lambda_i^2 X_{ii} = \lambda_i X_{ii}^2$, which implies $\lambda_i X_{ii} = X_{ii}^2$ (since $\lambda_i \neq 0$). The minimal polynomial of $X_{ii}$ must divide $x(x-\lambda_i)$, which means each $X_{ii}$ must be diagonalizable with $\sigma(X_{ii}) \subseteq \{0, \lambda_i\}$.

The equations for the remaining off-diagonal blocks become three independent, homogeneous Sylvester equations:
\begin{enumerate}
    \item For $X_{21}$: $\lambda_2 \lambda_1 X_{21} = \lambda_1 X_{21} X_{11} + \lambda_2 X_{22} X_{21}$
    \item For $X_{23}$: $\lambda_2 \lambda_3 X_{23} = \lambda_2 X_{22} X_{23} + \lambda_3 X_{23} X_{33}$
    \item For $X_{31}$: $\lambda_3 \lambda_1 X_{31} = \lambda_1 X_{31} X_{11} + \lambda_3 X_{33} X_{31}$
\end{enumerate}

Let us analyze the condition for a non-trivial solution for $X_{21}$. The equation can be rewritten as:
\[
(\lambda_2 X_{22} - \lambda_1 \lambda_2 I) X_{21} + \lambda_1 X_{21} X_{11} = 0.
\]
This is a homogeneous Sylvester equation of the form $AX - XB = 0$. A non-trivial solution for $X_{21}$ exists if and only if the spectra of the coefficient matrices, $\sigma(\lambda_2 X_{22} - \lambda_1 \lambda_2 I)$ and $\sigma(-\lambda_1 X_{11})$, have a common element.

The spectra are determined by the spectra of $X_{11}$ and $X_{22}$:
\begin{itemize}
    \item $\sigma(-\lambda_1 X_{11}) \subseteq \{0, -\lambda_1^2\}$
    \item $\sigma(\lambda_2 X_{22} - \lambda_1 \lambda_2 I) \subseteq \{-\lambda_1 \lambda_2, \lambda_2^2 - \lambda_1 \lambda_2\}$
\end{itemize}

Given that the eigenvalues $\lambda_1, \lambda_2$ are distinct and non-zero, an intersection can only occur if $\lambda_2^2 - \lambda_1 \lambda_2 = - \lambda_1^2$. This simplifies to the condition:
\[
\lambda_1^2 - \lambda_1 \lambda_2 + \lambda_2^2 = 0.
\]

If this condition on the eigenvalues holds, the corresponding Sylvester equation admits non-trivial solutions for $X_{21}$, which can then be fully characterized.

The analysis for $X_{23}$ and $X_{31}$ is perfectly analogous. Non-trivial solutions for these blocks exist if and only if the corresponding pairs of eigenvalues satisfy similar algebraic constraints:
\begin{itemize}
    \item For $X_{23}$: $\lambda_2^2 - \lambda_2 \lambda_3 + \lambda_3^2 = 0$.
    \item For $X_{31}$: $\lambda_3^2 - \lambda_3 \lambda_1 + \lambda_1^2 = 0$.
\end{itemize}

\end{document}
